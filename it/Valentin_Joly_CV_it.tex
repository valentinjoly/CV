\documentclass[letterpaper,12pt]{article}

% MARGINS
\usepackage[
    top=1.75cm,
    bottom=1.75cm,
    outer=2cm,
    inner=2cm,
    heightrounded,
    marginparwidth=3cm,
    marginparsep=0.5cm]{geometry}


% LANGUAGE
\usepackage{polyglossia}
\setdefaultlanguage{italian}


% FONTS
\usepackage{fontawesome}
\usepackage{fontspec}
\defaultfontfeatures{Mapping=tex-text}
\setmainfont{Lato}
\newfontfamily{\light}
  [Ligatures=TeX, UprightFont={* Light}, ItalicFont={* Light Italic},
   BoldFont={* Medium}, BoldItalicFont={* Medium Italic}]{Lato}
\newfontfamily{\heavy}
  [Ligatures=TeX, UprightFont={* Heavy}, ItalicFont={* Heavy Italic},
   BoldFont={* Heavy}, BoldItalicFont={* Heavy Italic}]{Lato}


% PAGE LAYOUT
\usepackage{parskip}
\usepackage{titlesec}
\titleformat{\section}{\large\heavy\raggedright}{}{0em}{}[\titlerule]
\titlespacing{\section}{0pt}{3pt}{3pt}


% COLOURS
\usepackage{xcolor}
\definecolor{linkcolour}{rgb}{0,0.2,0.6}


% HYPERLINKS
\usepackage{hyperref}
\hypersetup{
    colorlinks,
    breaklinks,
    urlcolor=linkcolour,
    linkcolor=linkcolour}

% TABLES
\usepackage{multirow}
\usepackage{tabularx}
\newcolumntype{R}{>{\raggedleft\arraybackslash}X}

\usepackage{fancyhdr}
\usepackage{lastpage}
\lhead{}\chead{}\rhead{}\lfoot{}\cfoot{}
\rfoot{\bfseries\small Pagina \thepage{} di \pageref*{LastPage}}
\renewcommand{\headrulewidth}{0pt}
\renewcommand{\footrulewidth}{0pt}
\footskip=5mm


% DOCUMENT
\begin{document}
\pagestyle{fancy}


% HEADER
\begin{tabularx}{\textwidth}{@{}lllR@{}}
  \multicolumn{3}{@{}l@{}}{\Large\heavy Valentin Joly}\vspace{1mm} & \multirow{4}{*}{\includegraphics[width=2.6cm]{../valentin.jpg}} \\
  \multicolumn{3}{@{}l@{}}{\large\bfseries Biologo molecolare • Bioinformatico}\vspace{4mm} &  \\
  \vspace{0.75mm} \faPhoneSquare~+1 (438) 495-3215
  & \faEnvelopeSquare~\href{mailto:valentin.joly@gmail.com}{valentin.joly@gmail.com}
  & \faLinkedinSquare~\href{https://www.linkedin.com/in/valentinjoly}{valentinjoly} & \\

  \faSkype~valentin.joly
  & \faExternalLinkSquare~\href{http://vjoly.net/it/index.html}{http://vjoly.net}
  & \faGithub~\href{https://github.com/valentinjoly}{valentinjoly} & \\
\end{tabularx}

\vspace{4mm}

{\light Il mio progetto di dottrato nel laboratorio Matton si concentra sui
meccanismi molecolari che regolano le barriere di isolamento prezigotico nelle
specie di patate selvatiche. Sono particolarmente interessato all’attrazione del
tubo pollinico, con un duplice approccio che coinvolge la bioinformatica
(sequenziamento next-gen) e la biologia molecolare (espressione proteica e
analisi funzionali).
\emph{Maggiori informazioni su \href{http://vjoly.net/it/index.html}{vjoly.net}.}}
\vspace{5mm}


% EDUCATION
\section{Formazione accademica}

\begin{tabularx}{\textwidth}{@{}l|X@{}}

  {\heavy Ph.D.}
  & {\heavy Dottorato in Scienze biologiche,} {\bfseries dal 2013}
    ~~~\small{(fine pianificato a giugno 2019)} \\

  {\heavy M.Sc.}
  & {\heavy Master in Scienze biologiche,} {\bfseries 2012}
    ~~~\small{(transizione accelerata al dottorato in 2013)} \vspace{0.5mm} \\
  & \hspace{1.5mm} Università di Montreal, \emph{Montreal, QC, Canada} \\
  & \hspace{1.5mm} {\small \textbf{Progetto:} Comunicazione molecolare tra gametofiti maschili e femminili} \\
  & \hspace{1.5mm} {\small \phantom{\textbf{Progetto:}} e barriere riproduttive nelle patate selvatiche (\emph{Solanum} sect. \emph{Petota}).} \\

  \multicolumn{2}{c}{} \\

  {\heavy B.Sc.}
  & {\heavy Laurea in Biologia, programma internazionale,} {\bfseries 2011} \vspace{0.5mm} \\
  & \hspace{1.5mm} Università Pierre e Marie Curie (UPMC),
    \emph{Parigi, Francia}: anni 1 e 2 \\
  & \hspace{1.5mm} Università di Montreal (UdeM),
    \emph{Montreal, QC, Canada}: anno 3 \\

\end{tabularx}

\vspace{6mm}


% RESEARCH EXPERIENCE
\section[Esperienza scientifica]{Esperienza scientifica
         \hfill \small{{\mdseries\faFlask}~Biologia molecolare~~~{\mdseries\faCode}~Bioinformatica}}

\begin{tabularx}{\textwidth}{@{}r|X@{}}

{\heavy Canada}
& {\heavy Progetto di dottorato,} {\bfseries dal 2013} \\
& {\em Prof. Daniel P. Matton, IRBV, Università di Montreal}
  \vspace{0.5mm} \\
& \small \hspace{1.5mm} \faFlask~Manipolazione di DNA e RNA. Clonazione. Espressione e purificazione di proteine. \\
& \small \hspace{1.5mm} \faFlask~Coltivazione di cellule vegetali. Test di attrazione di tubi pollinici. Microfluidica. \\
& \small \hspace{1.5mm} \faFlask~Microscopia: Epifluorescenza. Confocale. SEM. TEM. \\
& \small \hspace{1.5mm} \faCode~Programmazione Python e R. Sviluppo dello software KAPPA. \\
& \small \hspace{1.5mm} \faCode~Trascrittomica: Assemblaggio RNA-seq. Analisi di microarray. DGE. Annotazione. \\
& \small \hspace{1.5mm} \faCode~Proteomica: Analisi di dati LC-MS. Secretomica. Quantificazione \emph{label-free} delle proteine. \\

\multicolumn{2}{c}{} \\

{\heavy Svezia}
& {\heavy Collaborazione internazionale,} {\bfseries 2016--2018} \\
& {\em Dr. Johan Edqvist, Università di Linköping}
  \vspace{0.5mm} \\
& \small \hspace{1.5mm} \faFlask~Espressione e purificazione di proteine ​​in \emph{Pichia pastoris}. \\
& \small \hspace{1.5mm} \faCode~Sviluppo di un database e di uno strumento di predizione di nsLTP vegetali. \\

\multicolumn{2}{c}{} \\

{\heavy Giappone}
& {\heavy Programma estivo JSPS,} {\bfseries giugno–agosto 2016} \\
& {\em Prof. Tetsuya Higashiyama, ITbM, Università di Nagoya}
  \vspace{0.5mm} \\
& \small \hspace{1.5mm} \faFlask~Sviluppo di dispositivi microfluidici per test di attrazione di tubi pollinici. \\
& \small \hspace{1.5mm} \faFlask~Introduzione alla microscopia di eccitazione a due fotoni. \\

\end{tabularx}

\newpage

\section*{Esperienza scientifica \small{(continuazione)}
          \hfill \small{{\mdseries\faFlask}~Biologia molecolare~~~{\mdseries\faCode}~Bioinformatica}}

\begin{tabularx}{\textwidth}{@{}r|X@{}}

{\heavy Stati Uniti}
& {\heavy Stage internazionale di ricerca,} {\bfseries aprile–maggio 2014} \\
& {\em Prof. Willie J. Swanson, Università di Washington}
  \vspace{0.5mm} \\
& \small \hspace{1.5mm} \faCode~\emph{Variant calling} (GATK). \\
& \small \hspace{1.5mm} \faCode~Analisi dell'evoluzione molecolare e selezione positiva (codeml). \\

\multicolumn{2}{c}{} \\

{\heavy Argentina}
& {\heavy Transetto botanico,} {\bfseries aprile–maggio 2012} \\
& {\em Partnership con il Dr. Franco E. Chiarini, Università Nazionale di Córdoba}
  \vspace{0.5mm} \\
& \small \hspace{1.5mm} \faFlask~Raccolta di patate selvatiche nelle Ande. \\

\multicolumn{2}{c}{} \\

{\heavy Canada}
& {\heavy Stage di ricerca,} {\bfseries gennaio–agosto 2011} \\
& {\em Prof. Daniel P. Matton, Università di Montreal}
  \vspace{0.5mm} \\
& \small \hspace{1.5mm} \faFlask~Clonaggio molecolare. Biolistica. Epifluorescenza e microscopia confocale. \\

\multicolumn{2}{c}{} \\

{\heavy Francia}
& {\heavy Stage di ricerca,} {\bfseries giugno–luglio 2010} \\
& {\em Prof. Christophe Bailly, CNRS/Università Pierre e Marie Curie, Parigi}
  \vspace{0.5mm} \\
& \small \hspace{1.5mm} \faFlask~Biologia della dormienza e della germinazione dei semi. \vspace{2.5mm} \\
& {\heavy Stage di introduzione alla ricerca,} {\bfseries gennaio 2009} \\
& {\em Prof. Chris Bowler, CNRS/École Normale Supérieure, Parigi}
  \vspace{0.5mm} \\
& \small \hspace{1.5mm} \faFlask~Elettroforesi di proteine. Immunoprecipitazione. Western Blot. \\

\end{tabularx}

\vspace{6mm}


% EXTRA TRAINING
\section{Formazione complementare}

\begin{tabularx}{\textwidth}{@{}r|lX@{}}

\heavy{Bioinformatica}
& \multicolumn{2}{l}{{\heavy Specializzazione online in bioinformatica,} {\bfseries 2016--2018}} \\
& \multicolumn{2}{l}{\em Università di California San Diego, su Coursera \vspace{0.5mm}} \\

& \small \hspace{1.5mm} {\bfseries 1.} {\em Trovare messaggi nascosti nel DNA.}
& \small Certif. \href{https://www.coursera.org/account/accomplishments/verify/SPRUS2D6NH}{SPRUS2D6NH} \\

& \small \hspace{1.5mm} {\bfseries 2.} {\em Sequenziamento del genoma.}
& \small Certif. \href{https://www.coursera.org/account/accomplishments/verify/73HUUXBY64}{73HUUXBY64} \\

& \small \hspace{1.5mm} {\bfseries 3.} {\em Paragonare i geni, le proteine ed i genomi.}
& \small Certif. \href{https://www.coursera.org/account/accomplishments/verify/HY7JCN6UV2}{HY7JCN6UV2} \\

& \small \hspace{1.5mm} {\bfseries 4.} {\em Evoluzione molecolare.}
& \small Certif. \href{https://www.coursera.org/account/accomplishments/verify/VYKM2WT4792A}{VYKM2WT4792A} \\

& \small \hspace{1.5mm} {\bfseries 5.} {\em Scienza dei dati genomici e \emph{clustering}.}
& \small Certif. \href{https://www.coursera.org/account/accomplishments/verify/M6ZPV8VCEH}{M6ZPV8VCEH} \\

& \small \hspace{1.5mm} {\bfseries 6.} {\em Trovare mutazioni nel ADN e nelle proteine.}
& \small Certif. \href{https://www.coursera.org/account/accomplishments/verify/EVDAXLXV9L}{EVDAXLXV9L} \\

& \small \hspace{1.5mm} {\bfseries 7.} {\em Progetto finale: Il \emph{big data} in biologia.}
& \small Certif. \href{https://www.coursera.org/account/accomplishments/verify/56XJX7TMHYVM}{56XJX7TMHYVM} \\

& \small \hspace{1.5mm} {\bfseries Certificato globale della specializzazione.}
& \small Certif. \href{https://www.coursera.org/account/accomplishments/specialization/H528Q2K9KYB6}{H528Q2K9KYB6} \\

\multicolumn{2}{c}{} \\

\heavy{Python/R}
& \multicolumn{2}{l}{{\heavy Corsi online di bioinformatica,} {\bfseries 2016}} \\
& \multicolumn{2}{l}{\em Università Johns Hopkins, su Coursera \vspace{0.5mm}} \\

& \small \hspace{1.5mm} •~\emph{Python per la scienza dei dati genomici}.
& \small Certif. \href{https://www.coursera.org/account/accomplishments/verify/XHKWDB4XD7}{XHKWDB4XD7} \\

& \small \hspace{1.5mm} •~\emph{Introduzione alle tecnologie genomiche}.
& \small Certif. \href{https://www.coursera.org/account/accomplishments/verify/U88T89XKR2}{U88T89XKR2} \\

& \small \hspace{1.5mm} •~\emph{Programmazione con R}.
& \small Certif. \href{https://www.coursera.org/account/accomplishments/verify/X8NKEQAUU4}{X8NKEQAUU4} \\

\multicolumn{2}{c}{} \\

{\heavy Annotazione}
& \multicolumn{2}{l}{{\heavy Seminario internazionale sull’annotazione funzionale delle proteine,} {\bfseries 2012}} \\
{\heavy di sequenze}
& \multicolumn{2}{l}{\em BLAST2GO, Università di California Davis} \\

\end{tabularx}


\vspace{6mm}

% PUBLICATIONS
\section[Pubblicazioni]{Pubblicazioni \hfill \small{*Co-autori}}

\begin{tabularx}{\textwidth}{@{}r|X@{}}

2018
& Salminen TA, Eklund DM, \textbf{Joly V}, Blomqvist K, Matton DP
  e Edqvist J. (2018).
  Deciphering the evolution and development of the cuticle by studying lipid
  transfer proteins in mosses and liverworts.
  \emph{Plants}, 7(1), 6.
  DOI: \href{http://doi.org/10.3390/plants7010006}{10.3390/plants7010006}
  \\

\multicolumn{2}{c}{} \\

2015
& \textbf{Joly V} e Matton DP. (2015).
  KAPPA, a simple algorithm for the discovery and clustering of proteins defined
  by a key amino acid pattern.
  \emph{Bioinformatics}, 31(11), 1716--1723.
  DOI: \href{http://doi.org/10.1093/bioinformatics/btv047}
  {10.1093/bioinformatics/btv047}
  \vspace{3mm}
  \\

& Liu Y*, \textbf{Joly V*}, Dorion S, Rivoal J e Matton DP. (2015).
  The plant ovule secretome: a different view toward pollen-pistil interactions.
  \emph{Journal of Proteome Research}, 14(11):4763--75.
  DOI: \href{http://doi.org/10.1021/acs.jproteome.5b00618}
  {10.1021/acs.jproteome.5b00618}
  \vspace{3mm}
  \\

& Lafleur É*, Kapfer C*, \textbf{Joly V}, Liu Y, Tebbji F, Daigle C,
  Gray-Mitsumune M, Cappadocia M, Nantel A e Matton DP. (2015).
  The ScFRK1 MAPK kinase kinase (MAPKKK) from \emph{Solanum chacoense} is
  involved in embryo sac and pollen development.
  \emph{Journal of Experimental Botany}, 66(7), 1833--1843.
  DOI: \href{http://doi.org/10.1093/jxb/eru524}{10.1093/jxb/eru524}
  \\

\multicolumn{2}{c}{} \\

{\em prossime}
& \textbf{Joly V}, Tebbji F e Matton DP.
  Pollination type recognition from a distance by the ovary is revealed by a
  global transcriptomic analysis.
  {\bfseries\em Da sottomettere in ottobre 2018.}
  \vspace{3mm}
  \\

& \textbf{Joly V}, Liu Y e Matton DP.
  Comparative RNA-sequencing reveals female gameotphyte-sac specific transcripts
  in the \emph{frk1} embryo sac-less mutant from \emph{Solanum chacoense}.
  {\bfseries\em Da sottomettere in dicembre 2018.}
  \vspace{3mm}
  \\

& \textbf{Joly V} e Matton DP.
  A transcriptomic time-course reveals developmentally regulated transcripts
  during ovule genesis and maturation in \emph{Solanum chacoense}.
  {\bfseries\em Da sottomettere in marzo 2019.} \\

\end{tabularx}

\vspace{6mm}

\section[Codice informatico]{Codice informatico}

\begin{tabularx}{\textwidth}{@{}r|X@{}}

2015
& \textbf{Joly V} e Matton DP. Key Aminoacid Pattern-based Protein Analyzer
  (KAPPA). \\
& \small \hspace{1.5mm} •~Versione 1.1 pubblicata sotto licenza GPL su
  \href{https://github.com/valentinjoly/kappa-1.1}{GitHub}. \\
& \small \hspace{1.5mm} •~Versione 1.0 pubblicata sotto licenza GPL su
  \href{https://sourceforge.net/projects/kappa-sequence-search/}{SourceForge}.
  \\

\end{tabularx}

\vspace{6mm}

\section[Divulgazione scientifica]{Divulgazione scientifica}

\begin{tabularx}{\textwidth}{@{}r|X@{}}

2016
& \textbf{Joly V}. {\em Le sexe des plantes avec Valentin Joly.} Intervista
  radiofonica per il programma di divulgazione scientifica
  \href{http://ici.radio-canada.ca/emissions/les_annees_lumiere/2009-2010/chronique.asp?idChronique=404672}{\emph{Les années lumière}}
  su Radio Canada. Trasmesso il 24 aprile 2016. \\

\multicolumn{2}{c}{} \\

2014
& \textbf{Joly V}. {\em Les mots d’amour des plantes à fleurs}. Articolo scritto
  per \emph{L'ARN messager}, il giornale online degli studenti di biologia
  dell’Università di Montreal. Pubblicato il 19 dicembre 2014.
  \\

\end{tabularx}

\newpage

% ORAL PRESENTATIONS
\section[Presentazioni orali]{Presentazioni orali
         \small in conferenze scientifiche \hfill {\mdseries\faStar}~Premio}

\begin{tabularx}{\textwidth}{@{}r|X@{}}

2017
& \faStar~\textbf{Joly V}, Viallet C, Liu Y, Zaro A, Ceriotti F e Matton DP.
  \emph{Deciphering species-specific pollen tube guidance in \emph{Solanum}.}
  CSPB Eastern Regional Meeting, Montreal, QC, Canada;
  24–25 novembre 2017.
  \vspace{1.5mm}
  \\

& \textbf{Joly V}, Viallet C, Liu Y e Matton DP.
  \emph{Reproductive cysteine-rich proteins: key players in \emph{Solanum}
  speciation?}
  Plant Biology 2017, Honolulu, HI, Stati Uniti;
  23–28 giugno 2017.
  \\

\multicolumn{2}{c}{} \\

2015
& \faStar~\textbf{Joly V} e Matton DP.
  \emph{Plants’ secret words of love: rapid evolution of pollen–pistil
  recognition proteins drives reproductive isolation of wild potatoes.}
  Botany 2015, Edmonton, AB, Canada;
  26–29 luglio 2015.
  \vspace{1.5mm}
  \\

2013
& \faStar~\textbf{Joly V} e Matton DP.
  \emph{Comment éviter les liaisons dangereuses : secrets d’alcôve des pommes
  de terre.}
  Journées du Centre SÈVE, Wendake, QC, Canada;
  7–8 novembre 2013.
  \vspace{1.5mm}
  \\

& \faStar~\textbf{Joly V}, Liu Y e Matton DP.
  \emph{Divergence des protéines reproductives et maintien des barrières de
  spéciation chez les pommes de terre sauvages.}
  23\textsuperscript{e} Symposium des Sciences biologiques,
  Università di Montreal, Montreal, QC, Canada;
  21 marzo 2013.
  \\

\end{tabularx}

\vspace{6mm}

\section[Relatore invitato]{Presentazioni orali \small come relatore invitato}

\begin{tabularx}{\textwidth}{@{}r|X@{}}

2018
& \textbf{Joly V} e Matton DP.
  \emph{Potato sexomics: deciphering species-specific pollen tube guidance in
  wild potatoes with high-throughput sequencing technologies.}
  Dip. di biologia molecolare, cellulare e dello sviluppo,
  Università Yale, New Haven, CT, Stati Uniti;
  22 ottobre 2018.
  \\

\multicolumn{2}{c}{} \\

2016
& \textbf{Joly V} e Matton DP.
  \emph{Pollen tube guidance and reproductive isolation in wild potatoes.}
  Dip. di genomica funzionale,
  Università di Kanazawa, Giappone;
  18 agosto 2016.
  \vspace{1.5mm}
  \\

& \textbf{Joly V} e Matton DP.
  \emph{Species-specific pollen tube guidance in wild potatoes.}
  Laboratorio di biologia molecolare delle piante,
  Università di Kyoto, Giappone;
  12 agosto 2016.
  \vspace{1.5mm}
  \\

& \textbf{Joly V} e Matton DP.
  \emph{Deciphering potatoes’ words of love.}
  Institute for Transformative bio-Molecules (ITbM),
  Università di Nagoya, Giappone;
  13 luglio 2016.
  \\

\multicolumn{2}{c}{} \\

2015
& \textbf{Joly V} e Matton DP.
  \emph{Sex among wild potatoes: ladies wear the pants.}
  Centro di Genomica Strutturale e Funzionale,
  Università Concordia, Montreal, QC, Canada;
  16 luglio 2015.
  \\

\multicolumn{2}{c}{} \\

2014
& \textbf{Joly V} e Matton DP.
  \emph{Cell-cell communication between gametophytes and reproductive isolation
  in wild potatoes.}
  Dip. di Scienze genomiche, Università di Washington, Seattle, WA, Stati Uniti;
  24 aprile 2014.
  \\

\multicolumn{2}{c}{} \\

2013
& \textbf{Joly V} e Matton DP.
  \emph{Species-specificity of pollen-pistil interactions in wild potatoes.}
  Istituto di Genetica, Accademia delle Scienze di Cina, Pechino, Cina;
  24 ottobre 2013.
  \\

\end{tabularx}


% POSTER PRESENTATIONS
\section[Presentazioni con poster]{Presentazioni con poster
         \small in conferenze scientifiche \hfill {\mdseries\faStar}~Premio}

\begin{tabularx}{\textwidth}{@{}r|X@{}}

2018
& \textbf{Joly V} e Matton DP.
  \emph{Long-distance relationships: how the ovary perceives different
  pollination types at a distance.}
  Plant Biology 2018, Montreal, QC, Canada;
  14–18 luglio 2018.
  \\

\multicolumn{2}{c}{} \\

2016
& \faStar~\textbf{Joly V}, Liu Y, Dorion S, Rivoal J e Matton DP.
  \emph{Ovule secretomics reveal the importance of post-transcriptional
  regulation of reproductive proteins.}
  Plant Reproduction 2016, Tucson, AZ, Stati Uniti;
  18–23 marzo 2016.
  \vspace{1.5mm}
  \\

& \faStar~\textbf{Joly V} e Matton DP.
  \emph{KAPPA: exploring -omics data to detect and cluster cysteine-rich
  proteins.}
  [same conference as above]
  \\

\multicolumn{2}{c}{} \\

2015
& \faStar~\textbf{Joly V} e Matton DP.
  \emph{KAPPA: meeting the challenge of proteome-wide detection and clustering
  of cysteine-rich proteins.}
  High Performance Computing Symposium HPCS 2015, Montreal, QC, Canada;
  17–19 giugno 2015.
  \\

\multicolumn{2}{c}{} \\

2013
& \textbf{Joly V}, Liu Y e Matton DP.
  \emph{Interspecific divergence of reproductive proteins: the keystone of
  species-specific fertilization in wild potatoes?}
  10th Solanaceae Conference (SOL 2013), Pechino, Cina;
  13–17 ottobre 2013.
  \vspace{1.5mm}
  \\

& \textbf{Joly V} e Matton DP.
  \emph{Speciation genes in pollen-pistil interactions.}
  9th Canadian Plant Genomics Workshop, Halifax, NS, Canada;
  12–15 agosto 2013.
  \\

\end{tabularx}

\vspace{6mm}

% OTHER PRESENTATIONS
\section[Altre presentazioni]{Altre presentazioni \hfill \small{*Presentatore}}

\begin{tabularx}{\textwidth}{@{}r|X@{}}

2018
& \textbf{Joly V} e Matton DP*.
  \emph{Pre-zygotic barriers in inter-specific crosses: a leading role for small
  cysteine-rich protein attractant in wild potatoes species ?}
  Plant Biology 2018, Montreal, QC, Canada;
  14–18 luglio 2018.
  \\

\multicolumn{2}{c}{} \\

2017
& \textbf{Joly V} e Matton DP*.
  \emph{Pollination type recognition from a distance by the ovary is revealed
  by a global transcriptomic analysis.}
  5th International Symposium on Plant Signaling and Behavior, Matsue, Giappone;
  26 giugno – 1 luglio 2017.
  \\

\multicolumn{2}{c}{} \\

2013
& Liu Y*, Bai F, \textbf{Joly V} e Matton DP.
  \emph{Identification of female gametophyte-specific CRPs and isolation of
  pollen tube guidance attractant(s) in solanaceous species.}
  Journées du Centre SÈVE, Wendake, QC, Canada;
  7–8 novembre 2013.
  \vspace{1.5mm}
  \\

& Tebbji F, \textbf{Joly V} e Matton DP*. \emph{Pollination type recognition
  from a distance by the ovary is revealed by a global transcriptomic analysis.}
  10th Solanaceae Conference (SOL 2013), Pechino, Cina;
  13–17 ottobre 2013.
  \vspace{1.5mm}
  \\

& Liu Y*, \textbf{Joly V} e Matton DP.
  \emph{Isolation and characterization of the pollen tube attractant from}
  Solanum chacoense. [stessa conferenza]
  \\

\multicolumn{2}{c}{} \\

2011
& Daigle C*, \textbf{Joly V} e Matton DP.
  \emph{Discovering new MAPK signalling cascades involved in plant reproduction
  using co-expression analyses and deep transcriptomic sequencing of ovule
  and pollen tubes.}
  7th Canadian Plant Genomics Workshop, Niagara Falls, ON, Canada;
  22–25 agosto 2011.
  \\

\end{tabularx}

\newpage

% TEACHING
\section{Insegnamento}

\begin{tabularx}{\textwidth}{@{}r|X@{}}

{\heavy Fisiologia}
& {\heavy Capo assistente di insegnamento,} {\bfseries 2013--2018} \\
{\heavy vegetale}
& {\heavy Assistente di insegnamento,} {\bfseries 2011--2012} \\
& {\em Corsi pratici di fisiologia vegetale, Prof.~Jean Rivoal, Università di Montreal}
  \vspace{1mm} \\
& •~Carico di insegnamento: 140 ore per sessione, circa 80 studenti \\
& •~Corsi settimanali: una lezione (0:45) e una sessione pratica (2:30) \\
& •~Supervisione di 1-2 assistenti di insegnamento \\

\multicolumn{2}{c}{} \\

\heavy{Biologia}
& \heavy{Assistente di insegnamento,} {\bfseries 2014--2016} \\
\heavy{molecolare}
& {\em Corsi pratici di biologia molecolare, Prof.~D. P. Matton, Università di Montreal}
  \vspace{1mm} \\
& •~Carico di insegnamento: 110 ore per sessione, 10-20 studenti \\
\end{tabularx}

\vspace{6mm}

\section{Supervisione di tirocini}

\begin{tabularx}{\textwidth}{@{}r|llll@{}}
{\heavy M.Sc./Ph.D.}
 & \multicolumn{4}{X}{\small\em 

 Questi studenti latinoamericani sono stati ospitati nel laboratorio del mio professore come parte del Programma Leaders Emergenti nelle Americhe (ELAP-PFLA) organizzato dal governo del Canada. Sono stato il loro supervisore per tirocini di 5 a 6 mesi collegato al mio progetto di dottorato. \vspace{2mm}} \\
 & \textbf{• Kelly Rodrigues} & 2018-19 & Ph.D. & Univ. di San Paolo (Brasile) \\
 & \textbf{• Federico Ceriotti} & 2017-18 & M.Sc. & UN di Cuyo (Argentina) \\
 & \textbf{• Carlos Bravo} & 2016-17 & Ph.D. & UN Autonoma del Messico (Messico) \\
 & \textbf{• Laura González} & 2016 & Ph.D. & UN di Córdoba (Argentina) \\
 & \textbf{• Mariana Quiroga} & 2015 & Ph.D. & UN di Córdoba (Argentina) \\

\multicolumn{2}{c}{} \\

{\heavy B.Sc.}
 & \multicolumn{4}{X}{\small\em Ho supervisionato questi studenti di laurea per tirocini di 4 a 6 mesi richiesti nel loro programma accademico. \vspace{2mm}} \\
 & \textbf{• Maude Dorval} & 2018 & B.Sc. & Univ. di Montreal (Canada) \\
 & & 2017 & DEC & Collège Ahuntsic (Canada) \\
 & \textbf{• Anna Zaro Sánchez} & 2017 & B.Sc. & Univ. di Barcelona (Spagna) \\
 & \textbf{• Francis Banville} & 2017 & B.Sc. & Univ. di Montreal (Canada) \\
 & \textbf{• Andréa Davrinche} & 2014 &  B.Sc. & Univ. P. \& M. Curie (Francia) \\
 & \textbf{• Ella Gangbe} &  2013 & B.Sc. & Univ. di Montreal (Canada) \\
 & \textbf{• Tissicca Hour} &  2012 & B.Sc. & Univ. di Montreal (Canada) \\
\end{tabularx}

\newpage


% SCHOLARSHIPS AND AWARDS
\section[Premi e borse]{Premi e borse
         \hfill \small{{\mdseries\faStar}~Premio o borsa importante}}

\begin{tabularx}{\textwidth}{@{}r|X@{}}

2018

& \textbf{Borsa di viaggio Jacques-Rousseau} \\
& Istituto di Ricerca in Biologia Vegetale, Università di Montreal, 800~CAD \\

\multicolumn{2}{c}{} \\

2017

& \faStar~\textbf{Borsa di eccellenza Hydro-Québec (2ᵒ anno)} \\
& Hydro-Québec (compagnia elettrica nazionale), 25 000~CAD
  \vspace{1.3mm} \\

& \textbf{Borsa di fine di dottorato (BFED)} \\
& Facoltà di studi superiori e postdottorali, Università di Montreal, 8 400~CAD
  \vspace{1.3mm} \\

& \textbf{Borsa di viaggio Jacques-Rousseau} \\
& Istituto di Ricerca in Biologia Vegetale, Università di Montreal, 1 500~CAD
  \vspace{1.3mm} \\

& \textbf{Borsa di viaggio (\emph{Bourse d'appui à la diffusion des résultats de recherche})} \\
& Facoltà di studi superiori e postdottorali, Università di Montreal, 500~CAD
  \vspace{1.3mm} \\

& \textbf{Menzione d'onore per una presentazione orale (studenti)} \\
& CSPB Eastern Regional Meeting \\

\multicolumn{2}{c}{} \\

2016

& \faStar~\textbf{Borsa di eccellenza Hydro-Québec} \\
& Hydro-Québec (compagnia elettrica nazionale), 25 000~CAD
  \vspace{1.3mm} \\

& \faStar~\textbf{Borsa di dottorato} \\
& Governo del Québec (FRQNT), 13 333~CAD
  \vspace{1.3mm} \\

& \faStar~\textbf{Premio MITACS Globalink – Programma estivo della JSPS} \\
& MITACS / Japanese Society for the Promotion of Science, 534 000~JPY
  \vspace{1.3mm} \\

& \textbf{Premio per il miglior poster (studenti)} \\
& Frontiers in Plant Reproduction Biology, conferencia \emph{Plant Reproduction 2016}, 300~USD
  \vspace{1.3mm} \\

& \textbf{Borsa di viaggio Jacques-Rousseau} \\
& Istituto di Ricerca in Biologia Vegetale, Università di Montreal, 1 500~CAD
  \vspace{1.3mm} \\

& \textbf{Sussidio di viaggio PARSECS} \\
& FAÉCUM, Università di Montreal, 400~CAD \\

\multicolumn{2}{c}{} \\

2015

& \faStar~\textbf{Borsa di eccellenza Catherine-Frédette in scienze biologiche e neurologiche} \\
& Facoltà di studi superiori e postdottorali, Università di Montreal, 5 000~CAD
  \vspace{1.3mm} \\

& \faStar~\textbf{Borsa di dottorato FBSB del Dipartimento di Scienze Biologiche} \\
& Università di Montreal, 1 500~CAD
  \vspace{1.3mm} \\

& \textbf{President's Award per la migliore presentazione orale (studenti)} \\
& Società Canadese di Biologia Vegetale (CSPB-SCBV), Conferenza Botany 2015, 500~CAD
  \vspace{1.3mm} \\

& \textbf{Premio per la migliore presentazione orale (studenti)} \\
& Compute Canada, High Performance Computing Symposium HPCS 2015, 500~CAD
  \vspace{1.3mm} \\

& \textbf{Borsa di viaggio G.-H. Duff} \\
& Società Canadese di Biologia Vegetale (CSPB-SCBV), 340~CAD
  \vspace{1.3mm} \\

& \textbf{Borsa di viaggio Jacques-Rousseau} \\
& Istituto di Ricerca in Biologia Vegetale, Università di Montreal, 770~CAD
  \vspace{1.3mm} \\

& \faStar~\textbf{Borsa di eccellenza della Facoltà di studi superiori e postdottorali} \\
& Università di Montreal, 3 000~CAD \\

\end{tabularx}

\section*{Premi e borse \small{(continuazione)}
          \hfill \small{{\mdseries\faStar}~Premio o borsa importante}}

\begin{tabularx}{\textwidth}{@{}r|X@{}}

2014

& \faStar~\textbf{Borsa Pehr-Kalm} \\
& Orto Botanico di Montreal, 2 000~CAD
  \vspace{1.3mm} \\

& \textbf{Borsa di viaggio per stage internazionali} \\
& Governo del Québec (FRQNT) – Centre SÈVE, 3 815~CAD
  \vspace{1.3mm} \\

& \textbf{Borsa di viaggio Jacques-Rousseau} \\
& Istituto di Ricerca in Biologia Vegetale, Università di Montreal, 1 760~CAD \\

\multicolumn{2}{c}{} \\

2013

& \faStar~\textbf{Borsa di eccellenza Marie-Victorin} \\
& Istituto di Ricerca in Biologia Vegetale, Università di Montreal, 3 000~CAD
  \vspace{1.3mm} \\

& \textbf{Premio per la migliore presentazione orale} \\
& Journées du Centre SÈVE, 300~CAD
  \vspace{1.3mm} \\

& \textbf{Borsa di viaggio Jacques-Rousseau} \\
& Istituto di Ricerca in Biologia Vegetale, Università di Montreal, 850~CAD
  \vspace{1.3mm} \\

& \textbf{Premio per la migliore presentazione orale} \\
& Simposio di scienze biologiche, Università di Montreal, 100~CAD \\

\multicolumn{2}{c}{} \\

2012

& \faStar~\textbf{Borsa di master FBSB del Dipartimento di Scienze Biologiche} \\
& Università di Montreal, 1 200~CAD
  \vspace{1.3mm} \\

& \faStar~\textbf{Borsa di transizione accelerata master–dottorato} \\
& Facoltà di studi superiori e postdottorali, Università di Montreal, 14 000~CAD \\

\multicolumn{2}{c}{} \\

2011

& \textbf{Borsa di viaggio per uno scambio in Canada} \\
& Ministero francese della ricerca (CROUS), 1 600~EUR
  \vspace{1.3mm} \\

& \textbf{Borsa di eccellenza PIL per uno scambio in Canada} \\
& Università Pierre y Marie Curie (París VI), 1 500~EUR
  \vspace{1.3mm} \\

& \textbf{Borsa di viaggio AMIÉ per uno scambio in Canada} \\
& Autorità regionale di Parigi (\emph{Conseil régional}), 2 800~EUR
  \vspace{1.3mm} \\

& \textbf{Borsa di viaggio Campus'Trotter per uno scambio in Canada} \\
& Autorità locali in Francia (\emph{Conseil général}), 700~EUR \\

\multicolumn{2}{c}{} \\

2010

& \textbf{Miglior studente del dipartimento di biologia, esami di giugno 2010} \\
& Università Pierre e Marie Curie (Parigi VI) \\

\multicolumn{2}{c}{} \\

2008

& \faStar~\textbf{Borsa di eccellenza per studi universitari (laurea)} \\
& Ministero francese della ricerca (CROUS), 5 400~EUR \\

\end{tabularx}

\newpage

% COMMITMENTS
\section{Servizio}

\begin{tabularx}{\textwidth}{@{}r|X@{}}

{\heavy Società}

 & {\heavy Società Americana di Biologia Vegetale (ASPB),} {\bfseries dal 2016}
   \vspace{2mm} \\

 & {\heavy Società Canadese di Biologia Vegetale (SCBV-CSPB),} {\bfseries dal 2014}
   \vspace{2mm} \\

 & {\heavy Associazione Internazionale per la Ricerca} \\
 & {\heavy sulla Riproduzione Sessuale delle Piante (IASPRR),} {\bfseries dal 2015}
   \vspace{2mm} \\

 & {\heavy Associazioni dei Biologi del Quebec (ABQ),} {\bfseries 2013--2018}
   \vspace{2mm} \\

 & {\heavy Società Botanica di Francia (SBF),} {\bfseries 2010--2011}
   \\

\multicolumn{2}{c}{} \\

{\heavy Students'}
  & {\heavy Associazione degli studenti naturalisti \emph{Timarcha},} {\bfseries 2010--2011} \\
{\heavy associations}
  & Università Pierre e Marie Curie (UPMC), Parigi, Francia
    \vspace{2mm} \\

  & {\heavy Comitato di azioni ambientali \emph{Éco-école},} {\bfseries 2006--2008} \\
  & Lycée Saint-Sauveur ($\approx$ liceo), Redon, Francia \\

\multicolumn{2}{c}{} \\

{\heavy Volontariato}

 & {\heavy Insegnante volontario di francese per immigranti,} {\bfseries 2015--2016} \\
 & Centro communitario \emph{La Maison de l’Amitié}, Montreal, QC, Canada \\
 & •~Lezioni di 3 ore ogni settimana con 10-20 studenti
   \vspace{2mm} \\

 & {\heavy Collaboratore a vari progetti online:} \\
 & •~Scrittore e traduttore per \emph{Wikipedia}
   (articoli relativi alla biologia), dal 2008 \\
 & •~Cartografo volontario per \emph{OpenStreetMap}, dal 2015 \\
 & •~Digitalizzazione di erbari per il Museo Nazionale di Storia Naturale di Parigi (Progetto “\emph{Les Herbonautes}”), 2015 \\

\end{tabularx}

\vspace{6mm}

% OTHER SKILLS
\section{Altre competenze}

\begin{tabularx}{\textwidth}{@{}r|X@{}}

{\heavy Lingue}
& \textbf{Francese,} madrelingua \\
& \textbf{Inglese,} avanzato \\
& \textbf{Spagnolo,} avanzato \\
& \textbf{Italiano,} intermedio \\
& \textbf{Esperanto e Giapponese,} principiante \\

\multicolumn{2}{c}{} \\

{\heavy Informatica}
& \textbf{Programmazione:} Python e R. Base di C e Perl.
  \vspace{2mm} \\

& \textbf{Web:} HTML/CSS, Jekyll.
  \vspace{2mm} \\

& \textbf{Sistemi operativi:} Linux (\emph{Ubuntu}, \emph{Fedora},
  \emph{CentOS}), Mac OS X, Windows.
  \vspace{2mm} \\

& \textbf{Bioinformatica:} programmi di assemblaggio (\emph{Trinity}, \emph{CLC}, ecc.);
  allineatori di letture (\emph{Bowtie}, \emph{TopHat}, ecc.);
  strumenti di ricerca e di allineamento di sequenze (\emph{BLAST}, ecc.);
  annotatori (\emph{BLAST2GO}, \emph{PFAMscan}, \emph{SignalP}, ecc.)
  \vspace{2mm} \\

& \textbf{Programmi per l’ufficio:} \LaTeX, \emph{LibreOffice}/\emph{OpenOffice},
  \emph{Microsoft Office}
  \vspace{2mm} \\

& \textbf{Elaborazione delle immagini:} \emph{GIMP}, \emph{Inkscape}, \emph{ImageJ},
  \emph{Adobe Photoshop}, \emph{Cytoscape} ; \emph{AxioVision} (software di controllo dei microscopi \emph{Zeiss}) \\

\end{tabularx}

\end{document}
