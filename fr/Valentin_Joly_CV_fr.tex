\documentclass[letterpaper,12pt]{article}

% MARGINS
\usepackage[
    top=1.75cm,
    bottom=1.75cm,
    outer=2cm,
    inner=2cm,
    heightrounded,
    marginparwidth=3cm,
    marginparsep=0.5cm]{geometry}


% LANGUAGE
\usepackage{polyglossia}
\setdefaultlanguage{french}


% FONTS
\usepackage{fontawesome}
\usepackage{fontspec}
\defaultfontfeatures{Mapping=tex-text}
\setmainfont{Lato}
\newfontfamily{\light}
  [Ligatures=TeX, UprightFont={* Light}, ItalicFont={* Light Italic},
   BoldFont={* Medium}, BoldItalicFont={* Medium Italic}]{Lato}
\newfontfamily{\heavy}
  [Ligatures=TeX, UprightFont={* Heavy}, ItalicFont={* Heavy Italic},
   BoldFont={* Heavy}, BoldItalicFont={* Heavy Italic}]{Lato}


% PAGE LAYOUT
\usepackage{parskip}
\usepackage{titlesec}
\titleformat{\section}{\large\heavy\raggedright}{}{0em}{}[\titlerule]
\titlespacing{\section}{0pt}{3pt}{3pt}


% COLOURS
\usepackage{xcolor}
\definecolor{linkcolour}{rgb}{0,0.2,0.6}


% HYPERLINKS
\usepackage{hyperref}
\hypersetup{
    colorlinks,
    breaklinks,
    urlcolor=linkcolour,
    linkcolor=linkcolour}

% TABLES
\usepackage{multirow}
\usepackage{tabularx}
\newcolumntype{R}{>{\raggedleft\arraybackslash}X}

\usepackage{fancyhdr}
\usepackage{lastpage}
\lhead{}\chead{}\rhead{}\lfoot{}\cfoot{}
\rfoot{\bfseries\small Page \thepage{} sur \pageref*{LastPage}}
\renewcommand{\headrulewidth}{0pt}
\renewcommand{\footrulewidth}{0pt}
\footskip=5mm


% DOCUMENT
\begin{document}
\pagestyle{fancy}


% HEADER
\begin{tabularx}{\textwidth}{@{}llll@{}}
  \multicolumn{4}{@{}l@{}}{{\Large\heavy Valentin Joly,} {\Large Ph.D.}}\vspace{1mm} \\
  \multicolumn{4}{@{}l@{}}{\large\bfseries Chercheur postdoctoral en biologie moléculaire et bioinformatique}\vspace{4mm} \\

  \vspace{0.75mm}
    \faMapMarker~          Université Yale, É.-U.
  & \faEnvelopeSquare~     \href{mailto:valentin.joly@yale.edu}{valentin.joly@yale.edu}
  & \faExternalLinkSquare~ \href{http://vjoly.net/en/index.html}{http://vjoly.net}
  & \faLinkedinSquare~     \href{https://www.linkedin.com/in/valentinjoly}{valentinjoly} \\

  \vspace{0.75mm}
    \faFlag~               Français, Canadien
  & \faPhoneSquare~        +1 (475) 209-6054
  & \faSkype~              valentin.joly
  & \faGithub~             \href{https://github.com/valentinjoly}{valentinjoly} \\

\end{tabularx}

\vspace{4mm}

{\light
Au cours de mon doctorat au \href{https://www.irbv.umontreal.ca/chercheurs/daniel-philippe-matton}{labo Matton} de l’Université de Montréal, j’ai travaillé sur l’expression génique ovulaire et la communication pollen--pistil chez les pommes de terre sauvages, en utilisant la biologie moléculaire et la bioinformatique. En 2019, j’ai rejoint le \href{https://jacob-lab.yale.edu/}{\textbf{labo Jacob}} de l’Université Yale en tant que postdoc, avec un nouveau projet visant à révéler les fonctions cachées de l’hétérochromatine d'\emph{Arabidopsis} au moyen de CRISPR/Cas9. \textbf{\emph{Plus d’infos sur \href{http://vjoly.net/en/index.html}{vjoly.net}.}}}
\vspace{5mm}

% EDUCATION
\section{Formation académique}

\begin{tabularx}{\textwidth}{@{}l|X@{}}

  {\heavy Ph.D.}
  & {\heavy Sciences biologiques,} {\bfseries 2019}
    ~~~\small{(mention \emph{Exceptionnel})} \\

  {\heavy M.Sc.}
  & {\heavy Sciences biologiques,} {\bfseries 2012}
    ~~~\small{(passage accéléré au doctorat en 2013)} \vspace{0.5mm} \\
  & \hspace{1.5mm} Université de Montréal, \emph{Montréal, QC, Canada} \vspace{0.5mm} \\
  & \hspace{1.5mm} {\small \textbf{Directeur de recherche :} P\textsuperscript{r} Daniel P. Matton, Institut de Recherche en Biologie Végétale (IRBV)} \\
  & \hspace{1.5mm} {\small \textbf{Thèse:} Exploration bioinformatique des interactions pollen--pistil chez \emph{Solanum chacoense}.} \\

  \multicolumn{2}{c}{} \\

  {\heavy B.Sc.}
  & {\heavy Sciences du vivant, programme international,} {\bfseries 2011} \vspace{0.5mm} \\
  & \hspace{1.5mm} Université Pierre et Marie Curie (UPMC),
    \emph{Paris, France}: 1\textsuperscript{re} et 2\textsuperscript{e} années \\
  & \hspace{1.5mm} Université de Montréal (UdeM),
    \emph{Montréal, QC, Canada}: 3\textsuperscript{e} année en échange \\
\end{tabularx}

\vspace{5mm}


% RESEARCH EXPERIENCE
\section[Expérience en recherche]{Expérience en recherche
         \hfill \small{{\mdseries\faFlask}~Biologie moléculaire~~~{\mdseries\faCode}~Bioinformatique}}

\begin{tabularx}{\textwidth}{@{}r|X@{}}

{\heavy États-Unis}
& {\heavy Chercheur postdoctoral,} {\bfseries depuis 2019} \\
& {\em P\textsuperscript{r} Yannick Jacob, MCDB, Université Yale}
  \vspace{0.5mm} \\
& \small \hspace{1.5mm} \faFlask~Édition du génome par CRISPR/Cas9.\\
& \small \hspace{1.5mm} \faCode~Transcriptomique et méthylomique à haut débit. \\

\multicolumn{2}{c}{} \\

{\heavy Canada}
& {\heavy Projet de doctorat,} {\bfseries 2013--2019} \\
& {\em P\textsuperscript{r} Daniel P. Matton, IRBV, Université de Montréal}
  \vspace{0.5mm} \\
& \small \hspace{1.5mm} \faFlask~Manipulation d'ADN et ARN. Clonage. Expression et purification de protéines. \\
& \small \hspace{1.5mm} \faFlask~Cultures cellulaires végétales. Tests de guidage du tube pollinique. Microfluidique. \\
& \small \hspace{1.5mm} \faFlask~Microscopie: épifluorescence, confocal, MEB, MET. \\
& \small \hspace{1.5mm} \faCode~Python et R. Développement de l'outil de recherche de séquences KAPPA. \\
& \small \hspace{1.5mm} \faCode~Transcriptomique: Assemblages RNA-seq. Biopuces. DGE. Annotation fonctionnelle. \\
& \small \hspace{1.5mm} \faCode~Protéomique: Analyse de données LC-MS. Sécrétomique. Quantification \emph{label-free}. \\

\multicolumn{2}{c}{} \\

{\heavy Suède}
& {\heavy Collaboration internationale,} {\bfseries 2016--2018} \\
& {\em D\textsuperscript{r} Johan Edqvist, Université de Linköping}
  \vspace{0.5mm} \\
& \small \hspace{1.5mm} \faFlask~Expression et purification de protéines chez \emph{Pichia pastoris}. \\
& \small \hspace{1.5mm} \faCode~Développement d'un outil de prédiction et d'une base de données de nsLTP. \\

\end{tabularx}

\newpage

\section*{Expérience en recherche \small{(suite)}
          \hfill \small{{\mdseries\faFlask}~Biologie moléculaire~~~{\mdseries\faCode}~Bioinformatique}}

\begin{tabularx}{\textwidth}{@{}r|X@{}}

{\heavy Japon}
& {\heavy Programme d’été de la JSPS,} {\bfseries juin–août 2016} \\
& {\em P\textsuperscript{r} Tetsuya Higashiyama, ITbM, Université de Nagoya}
  \vspace{0.5mm} \\
& \small \hspace{1.5mm} \faFlask~Dispositifs microfluidiques pour l’étude du guidage des tubes polliniques. \\
& \small \hspace{1.5mm} \faFlask~Introduction à la microscopie confocale à deux photons. \\

\multicolumn{2}{c}{} \\

{\heavy États-Unis}
& {\heavy Stage international de recherche,} {\bfseries avr.–mai 2014} \\
& {\em P\textsuperscript{r} Willie J. Swanson, University of Washington}
  \vspace{0.5mm} \\
& \small \hspace{1.5mm} \faCode~Détection de variants dans des données de séquençage de masse (GATK). \\
& \small \hspace{1.5mm} \faCode~Étude de l’évolution moléculaire des séquences (sélection positive) avec codeml. \\

\multicolumn{2}{c}{} \\

{\heavy Argentine}
& {\heavy Séjour botanique sur le terrain,} {\bfseries avr.–mai 2012} \\
& {\em Partenariat avec le D\textsuperscript{r} Franco E. Chiarini, Universidad Nacional de Córdoba}
  \vspace{0.5mm} \\
& \small \hspace{1.5mm} \faFlask~Collecte d’individus de pommes de terre sauvages dans la cordillère des Andes. \\

\multicolumn{2}{c}{} \\

{\heavy Canada}
& {\heavy Stage de recherche,} {\bfseries janv.–août 2011} \\
& {\em P\textsuperscript{r} Daniel P. Matton, Université de Montréal}
  \vspace{0.5mm} \\
& \small \hspace{1.5mm} \faFlask~Clonage moléculaire. Biolistique. Microscopie confocale et à épifluorescence. \\

\multicolumn{2}{c}{} \\

{\heavy France}
& {\heavy Stage de recherche,} {\bfseries juin–juill. 2010} \\
& {\em P\textsuperscript{r} Christophe Bailly, CNRS/Université Pierre et Marie Curie, Paris}
  \vspace{0.5mm} \\
& \small \hspace{1.5mm} \faFlask~Physiologie de la dormance des semences.
  \vspace{2.5mm} \\
& {\heavy Stage court d’initiation à la recherche,} {\bfseries janv. 2009} \\
& {\em P\textsuperscript{r} Chris Bowler, CNRS/École Normale Supérieure, Paris}
  \vspace{0.5mm} \\
& \small \hspace{1.5mm} \faFlask~Électrophorèse de protéines. Immunoprécipitation. Western Blot. \\

\end{tabularx}

\vspace{6mm}


% EXTRA TRAINING
\section{Formation complémentaire}

\begin{tabularx}{\textwidth}{@{}r|X@{}}

\heavy{Bioinformatique}
& {\heavy Spécialisation en ligne en bioinformatique,} {\bfseries 2016--2018} \\
& \em University of California San Diego, sur Coursera \vspace{0.5mm} \\

& {\small \textbf{1.}~{\em Trouver les messages cachés dans l’ADN.} \textbf{2.}~{\em Séquençage des génomes.} \textbf{3.}~{\em Comparer les gènes, les protéines et les génomes.} \textbf{4.}~{\em Évolution moléculaire.} \textbf{5.}~{\em Science des données génomiques et \emph{clustering}.} \textbf{6.}~{\em Trouver les mutations dans l’ADN et les protéines.} \textbf{7.}~{\em Projet final : Le \emph{big data} en biologie.}} \\
& {\small {\bfseries Certificat global pour la spécialisation:}~\href{https://www.coursera.org/account/accomplishments/specialization/H528Q2K9KYB6}{H528Q2K9KYB6}} \\

\multicolumn{2}{c}{} \\

\heavy{Python et R}
& {\heavy Cours en ligne de bioinformatique,} {\bfseries 2016} \\
& \em Johns Hopkins University, sur Coursera \vspace{0.5mm} \\

& \small •~\emph{Python pour la science des données génomiques.} \textbf{Certificat :}~\href{https://www.coursera.org/account/accomplishments/verify/XHKWDB4XD7}{XHKWDB4XD7} \\

& \small •~\emph{Introduction aux technologies génomiques.} \textbf{Certificat :}~\href{https://www.coursera.org/account/accomplishments/verify/U88T89XKR2}{U88T89XKR2} \\

& \small •~\emph{Programmation en R.} \textbf{Certificat :}~\href{https://www.coursera.org/account/accomplishments/verify/X8NKEQAUU4}{X8NKEQAUU4} \\

\multicolumn{2}{c}{} \\

{\heavy Annotation}
& {\heavy Séminaire international sur l’annotation fonctionnelle des protéines,} {\bfseries 2012} \\
{\heavy de séquences}
& \em BLAST2GO, University of California Davis \\

\end{tabularx}

\vspace{6mm}

% PUBLICATIONS
\section[Publications]{Publications \hfill \small{*Contributions égales}}

\begin{tabularx}{\textwidth}{@{}r|X@{}}

2019
& \textbf{Joly V*}, Tebbji F*, Nantel A et Matton DP.
  Pollination type recognition from a distance by the ovary is revealed by a
  global transcriptomic analysis.
  \emph{Plants}, 2019, 8(6), 185.
  DOI: \href{http://doi.org/10.3390/plants8060185}{10.3390/plants8060185}
  \vspace{3mm}
  \\

& Mazin BD, \textbf{Joly V} et Matton DP. (2019).
  The ScFRK2 mitogen-activated protein kinase kinase kinase (MAP3K) is involved
  in early embryo sac development in \emph{Solanum chacoense}.
  \emph{Plant Signaling \& Behavior}, 14(8), 1620059.
  DOI: \href{http://doi.org/10.1080/15592324.2019.1620059}
  {10.1080/15592324.2019.1620059}
  \\

\multicolumn{2}{c}{} \\

2018
& Salminen TA, Eklund DM, \textbf{Joly V}, Blomqvist K, Matton DP
  et Edqvist J. (2018).
  Deciphering the evolution and development of the cuticle by studying lipid
  transfer proteins in mosses and liverworts.
  \emph{Plants}, 7(1), 6.
  DOI: \href{http://doi.org/10.3390/plants7010006}{10.3390/plants7010006}
  \\

\multicolumn{2}{c}{} \\

2015
& \textbf{Joly V} et Matton DP. (2015).
  KAPPA, a simple algorithm for the discovery and clustering of proteins defined by
  a key amino acid pattern.
  \emph{Bioinformatics}, 31(11), 1716--1723.
  DOI: \href{http://doi.org/10.1093/bioinformatics/btv047}
  {10.1093/bioinformatics/btv047}
  \vspace{3mm}
  \\

& Liu Y*, \textbf{Joly V*}, Dorion S, Rivoal J et Matton DP. (2015).
  The plant ovule secretome: a different view toward pollen-pistil interactions.
  \emph{Journal of Proteome Research}, 14(11):4763--75.
  DOI: \href{http://doi.org/10.1021/acs.jproteome.5b00618}
  {10.1021/acs.jproteome.5b00618}
  \vspace{3mm}
  \\

& Lafleur É*, Kapfer C*, \textbf{Joly V}, Liu Y, Tebbji F, Daigle C,
  Gray-Mitsumune M, Cappadocia M, Nantel A et Matton DP. (2015).
  The ScFRK1 MAPK kinase kinase (MAPKKK) from \emph{Solanum chacoense} is
  involved in embryo sac and pollen development.
  \emph{Journal of Experimental Botany}, 66(7), 1833--1843.
  DOI: \href{http://doi.org/10.1093/jxb/eru524}{10.1093/jxb/eru524}
  \\

%\multicolumn{2}{c}{} \\
%
%{\em soumises}
%& 
%  \\

\multicolumn{2}{c}{} \\

{\em à venir}
& \textbf{Joly V*}, Liu Y* and Matton DP.
  Transcriptomic profiling of \emph{Solanum chacoense} mature, immature, and
  embryo sac-less ovules.
  {\bfseries\em Soumission prévue en septembre 2019.}
  \\

\end{tabularx}

\vspace{6mm}

\section[Code informatique]{Code informatique}

\begin{tabularx}{\textwidth}{@{}r|X@{}}

2015
& \textbf{Joly V} et Matton DP. Key Aminoacid Pattern-based Protein Analyzer (KAPPA). \\
& \small \hspace{1.5mm} \textbullet{}~Version 1.1 publiée sous licence GPL sur la plate-forme \href{https://github.com/valentinjoly/kappa-1.1}{GitHub}. \\
& \small \hspace{1.5mm} \textbullet{}~Version 1.0 publiée sous licence GPL sur la plate-forme \href{https://sourceforge.net/projects/kappa-sequence-search/}{SourceForge}.
\\

\end{tabularx}

\newpage

% ORAL PRESENTATIONS
\section[Présentations orales]{
  Présentations orales \small dans des congrès scientifiques \hfill {\mdseries\faStar}~Prix}

\begin{tabularx}{\textwidth}{@{}r|X@{}}

2017
& \faStar~\textbf{Joly V}, Viallet C, Liu Y, Zaro A, Ceriotti F et Matton DP.
  \emph{Deciphering species-specific pollen tube guidance in \emph{Solanum}.}
  Rencontres régionales de l’Est du Canada, SCBV, Montréal, QC, Canada,
  24--25 nov. 2017.
  \vspace{1.5mm}
  \\

& \textbf{Joly V}, Viallet C, Liu Y et Matton DP.
  \emph{Reproductive cysteine-rich proteins: key players in \emph{Solanum}
  speciation?}
  Plant Biology 2017, Honolulu, HI, É.-U.,
  23--28 juin 2017.
  \\

\multicolumn{2}{c}{} \\

2015
& \faStar~\textbf{Joly V} et Matton DP.
  \emph{Plants’ secret words of love: rapid evolution of pollen–pistil
  recognition proteins drives reproductive isolation of wild potatoes.}
  Botany 2015, Edmonton, AB, Canada,
  26--29 juill. 2015.
  \\

\multicolumn{2}{c}{} \\

2013
& \faStar~\textbf{Joly V} et Matton DP.
  \emph{Comment éviter les liaisons dangereuses : secrets d’alcôve des pommes
  de terre.}
  Journées du Centre SÈVE, Wendake, QC, Canada,
  7--8 nov. 2013.
  \vspace{1.5mm}
  \\

& \faStar~\textbf{Joly V}, Liu Y et Matton DP.
  \emph{Divergence des protéines reproductives et maintien des barrières de
  spéciation chez les pommes de terre sauvages.}
  23\textsuperscript{e} Symposium des Sciences biologiques,
  Université de Montréal, Montréal, QC, Canada,
  21 mars 2013.
  \\

\end{tabularx}

\vspace{6mm}

\section[Conférencier invité]{Présentations orales \small comme conférencier invité}

\begin{tabularx}{\textwidth}{@{}r|X@{}}

2018
& \textbf{Joly V} et Matton DP.
  \emph{Potato sexomics: deciphering species-specific pollen tube guidance in
  wild potatoes with high-throughput sequencing technologies.}
  Dép. de Biologie moléculaire, cellulaire et du développement,
  Université Yale, New Haven, CT, États-Unis,
  22 oct. 2018.
  \\

\multicolumn{2}{c}{} \\

2016
& \textbf{Joly V} et Matton DP.
  \emph{Pollen tube guidance and reproductive isolation in wild potatoes.}
  Dép. de Génomique fonctionnelle,
  Université de Kanazawa, Japon,
  18 août 2016.
  \vspace{1.5mm}
  \\

& \textbf{Joly V} et Matton DP.
  \emph{Species-specific pollen tube guidance in wild potatoes.}
  Laboratoire de Biologie moléculaire des plantes,
  Université de Kyoto, Japon,
  12 août 2016.
  \vspace{1.5mm}
  \\

& \textbf{Joly V} et Matton DP.
  \emph{Deciphering potatoes’ words of love.}
  Institute for Transformative bio-Molecules (ITbM),
  Université de Nagoya, Japon,
  13 juill. 2016.
  \\

\multicolumn{2}{c}{} \\

2015
& \textbf{Joly V} et Matton DP.
  \emph{Sex among wild potatoes: ladies wear the pants.}
  Centre de Génomique Structurale et Fonctionnelle, Université Concordia,
  Montréal, QC, Canada,
  16 juill. 2015.
  \\

\multicolumn{2}{c}{} \\

2014
& \textbf{Joly V} et Matton DP.
  \emph{Cell-cell communication between gametophytes and reproductive
  isolation in wild potatoes.}
  Dept. of Genome Sciences, University of Washington, Seattle, WA, É.-U.,
  24 avr. 2014.
  \\

\multicolumn{2}{c}{} \\

2013
& \textbf{Joly V} et Matton DP.
  \emph{Species-specificity of pollen-pistil interactions in wild potatoes.}
  Institut de Génétique, Académie des Sciences de Chine, Pékin, Chine,
  24 oct. 2013.
  \\

\end{tabularx}


% POSTER PRESENTATIONS
\section[Présentations par affiche]{Présentations par affiche \small dans des congrès scientifiques \hfill {\mdseries\faStar}~Prix}

\begin{tabularx}{\textwidth}{@{}r|X@{}}

2018

& \textbf{Joly V} et Matton DP.
  \emph{Long-distance relationships: how the ovary perceives different
  pollination types at a distance.}
  Plant Biology 2018, Montréal, QC, Canada,
  14--18 juill. 2018.
  \\

\multicolumn{2}{c}{} \\

2016
& \faStar~\textbf{Joly V}, Liu Y, Dorion S, Rivoal J et Matton DP.
  \emph{Ovule secretomics reveal the importance of post-transcriptional
  regulation of reproductive proteins.}
  Plant Reproduction 2016, Tucson, AZ, É.-U.,
  18--23 mars 2016.
  \vspace{1.5mm} \\

& \faStar~\textbf{Joly V} et Matton DP.
  \emph{KAPPA: exploring -omics data to detect and cluster cysteine-rich
  proteins.}
  [même conférence que ci-dessus]
  \\

\multicolumn{2}{c}{} \\

2015
& \faStar~\textbf{Joly V} et Matton DP.
  \emph{KAPPA: meeting the challenge of proteome-wide detection and clustering
  of cysteine-rich proteins.}
  High Performance Computing Symposium HPCS 2015, Montréal, QC, Canada,
  17--19 juin 2015.
  \\

\multicolumn{2}{c}{} \\

2013
& \textbf{Joly V}, Liu Y et Matton DP.
  \emph{Interspecific divergence of reproductive proteins: the keystone of
  species-specific fertilization in wild potatoes?}
  10th Solanaceae Conference (SOL 2013), Pékin, Chine,
  13--17 oct. 2013.
  \vspace{1.5mm} \\

& \textbf{Joly V} et Matton DP.
  \emph{Speciation genes in pollen-pistil interactions.}
  9th Canadian Plant Genomics Workshop, Halifax, NS, Canada,
  12--15 août 2013.
  \\

\end{tabularx}

\vspace{6mm}

% OTHER PRESENTATIONS
\section[Autres présentations]{Autres présentations
\hfill \small{*Personne en charge}}

\begin{tabularx}{\textwidth}{@{}r|X@{}}

2019
& Mazin BD*, Daigle C, \textbf{Joly V} and Matton DP.
  \emph{The ScFRK2 and ScFRK3 MAP Kinase Kinase Kinase are involved in ovule
  development in \emph{Solanum chacoense}.}
  Plant Biology 2019, San Jose, CA, Canada;
  Aug. 3--7, 2019.
  \\

\multicolumn{2}{c}{} \\

2018
& \textbf{Joly V} et Matton DP*.
  \emph{Pre-zygotic barriers in inter-specific crosses: a leading role for small
  cysteine-rich protein attractant in wild potatoes species ?}
  Plant Biology 2018, Montréal, QC, Canada, 14--18 juill. 2018.
  \\

\multicolumn{2}{c}{} \\

2017
& \textbf{Joly V} et Matton DP*.
  \emph{Pollination type recognition from a distance by the ovary is revealed
  by a global transcriptomic analysis.}
  5th International Symposium on Plant Signaling and Behavior, Matsue, Japon,
  26 juin -- 1\textsuperscript{er} juill. 2017.
  \\

\multicolumn{2}{c}{} \\

2013
& Liu Y*, Bai F, \textbf{Joly V} et Matton DP.
  \emph{Identification of female gametophyte-specific CRPs and isolation of
  pollen tube guidance attractant(s) in solanaceous species.}
  Journées du Centre SÈVE, Wendake, QC, Canada, 7--8 nov. 2013.
  \vspace{1.5mm} \\

& Tebbji F, \textbf{Joly V} et Matton DP*. \emph{Pollination type recognition
  from a distance by the ovary is revealed by a global transcriptomic analysis.}
  10th Solanaceae Conference (SOL 2013), Pékin, Chine, 13--17 oct. 2013.
  \vspace{1.5mm} \\

& Liu Y*, \textbf{Joly V} et Matton DP.
  \emph{Isolation and characterization of the pollen tube attractant from}
  Solanum chacoense. [même conférence que ci-dessus] \\

\multicolumn{2}{c}{} \\

2011
& Daigle C*, \textbf{Joly V} et Matton DP.
  \emph{Discovering new MAPK signalling cascades involved in plant reproduction
  using co-expression analyses and deep transcriptomic sequencing of ovule
  and pollen tubes.}
  7th Canadian Plant Genomics Workshop, Niagara Falls, ON, Canada,
  22--25 août 2011.
  \\

\end{tabularx}

\vspace{6mm}

\section[Vulgarisation]{Œuvres de vulgarisation}

\begin{tabularx}{\textwidth}{@{}r|X@{}}

2016 & \textbf{Joly V}. {\em Le sexe des plantes avec Valentin Joly.} Entrevue radiophonique pour la rubrique {\em Doc/Post-doc} de \href{http://ici.radio-canada.ca/emissions/les_annees_lumiere/2015-2016/chronique.asp?idChronique=404672}{\em Les années lumière}, l’émission scientifique de la radio française de Radio-Canada. Diffusé le 24 avril 2016. \\

\multicolumn{2}{c}{} \\

2014 & \textbf{Joly V}. {\em Les mots d’amour des plantes à fleurs.} Article rédigé pour {\em L’ARN messager}, journal des étudiants en biologie de l’Université de Montréal. Publié le 19 décembre 2014. \\

\end{tabularx}

\newpage

% TEACHING
\section{Enseignement}
\begin{tabularx}{\textwidth}{@{}r|X@{}}

{\heavy Physiologie}
& {\heavy Assistant d’enseignement en chef,} {\bfseries 2013--2018} \\
{\heavy végétale}
& {\heavy Assistant d’enseignement,} {\bfseries 2011--2012} \\
& {\em TP de physiologie végétale, P\textsuperscript{r}~Jean Rivoal, Univ. de Montréal} \vspace{1mm} \\
& \textbullet{} Charge: 140 heures par session, environ 70 étudiants \\
& \textbullet{} Séances hebdomadaires incluant un laïus (45 min) et des travaux
  pratiques \\
& \textbullet{} Encadrement de 1 à 2 auxiliaires d'enseignement \\

\multicolumn{2}{c}{} \\

\heavy{Biologie}
& \heavy{Assistant d’enseignement,} {\bfseries 2014--2016} \\
\heavy{moléculaire}
& {\em TP de biologie moléculaire : ADN et ARN, P\textsuperscript{r}~Daniel P. Matton, Univ. de Montréal} \vspace{1mm} \\
& \textbullet{}~Enseignement de travaux pratiques à un groupe de 10 à 20
  étudiants \\
\end{tabularx}

\vspace{6mm}

\section{Supervision de stagiaires}

\begin{tabularx}{\textwidth}{@{}r|llll@{}}
{\heavy Cycles supérieurs}
 & \multicolumn{4}{X}{\small\em Ces étudiant·e·s d’Amérique latine ont travaillé
 dans le laboratoire de mon directeur de recherche, Daniel P. Matton, dans le
 cadre du Programme des Futurs Leaders dans les Amériques (PFLA) du Gouvernement
 du Canada. J’ai été leur superviseur direct pour des stages de 5 à 6 mois en
 lien avec mon projet de doctorat. \vspace{2mm}} \\
 & \textbf{• Kelly Rodrigues} & 2018-19 & Ph.D. & Univ. de São Paulo (Brésil) \\
 & \textbf{• Federico Ceriotti} & 2017-18 & M.Sc. & UN de Cuyo (Argentine) \\
 & \textbf{• Carlos Bravo} & 2016-17 & Ph.D. & UN du Mexique (Mexique) \\
 & \textbf{• Laura González} & 2016 & Ph.D. & UN de Córdoba (Argentine) \\
 & \textbf{• Mariana Quiroga} & 2015 & Ph.D. & UN de Córdoba (Argentine) \\

\multicolumn{2}{c}{} \\

{\heavy Premier cycle}
 & \multicolumn{4}{X}{\small\em J’ai supervisé ces étudiant·e·s pour des stages
 de 4 à 6 mois crédités et nécessaires à l’obtention de leur diplôme.
 \vspace{2mm}} \\
 & \textbf{• Maude Dorval} & 2018 & B.Sc. & Univ. de Montréal (Canada) \\
 & & 2017 & DEC & Collège Ahuntsic (Canada) \\
 & \textbf{• Anna Zaro Sánchez} & 2017 & B.Sc. & Univ. de Barcelone (Espagne) \\
 & \textbf{• Francis Banville} & 2017 & B.Sc. & Univ. de Montréal (Canada) \\
 & \textbf{• Andréa Davrinche} & 2014 &  B.Sc. & Univ. P. et M. Curie (France) \\
 & \textbf{• Ella Gangbe} &  2013 & B.Sc. & Univ. de Montréal (Canada) \\
 & \textbf{• Tissicca Hour} &  2012 & B.Sc. & Univ. de Montréal (Canada) \\
\end{tabularx}

\newpage

% SCHOLARSHIPS AND AWARDS
\section[Prix et bourses]{Prix et bourses
         \hfill \small{{\mdseries\faStar}~Bourse ou prix important}}

\begin{tabularx}{\textwidth}{@{}r|X@{}}

2019--21

& \faStar~\textbf{Bourse de recherche postdoctorale (B3X)} \\
& Fonds de Recherche du Québec -- Nature et Technologies (FRQNT) \\
& Gouvernement du Québec, Canada, 110~000~CAD \\

\multicolumn{2}{c}{} \\

2018

& {\heavy Bourse de voyage Jacques-Rousseau} \\
& Institut de Recherche en Biologie Végétale, Université de Montréal, 800~CAD \\

\multicolumn{2}{c}{} \\

2017

& \faStar~{\heavy Bourse d'excellence Hydro-Québec} \\
& Hydro-Québec (compagnie nationale d'électricité), 25~000~CAD
  \vspace{1.3mm} \\

& {\heavy Bourse de fin d'études doctorales (5\textsuperscript{e} année)} \\
& Faculté des Études Supérieures et Postdoctorales, Univ. de Montréal, 12~000~CAD \\

& {\heavy Bourse de voyage Jacques-Rousseau} \\
& Institut de Recherche en Biologie Végétale, Université de Montréal, 1~500~CAD
  \vspace{1.3mm} \\

& {\heavy Bourse d'appui à la diffusion des résultats de recherche} \\
& Faculté des Études Supérieures et Postdoctorales, Univ. de Montréal, 500~CAD
  \vspace{1.3mm} \\

& {\heavy Mention honorable} pour une présentation orale étudiante \\
& Rencontres régionales de l’Est du Canada, SCBV \\

\multicolumn{2}{c}{} \\

2016

& \faStar~{\heavy Bourse d'excellence Hydro-Québec} \\
& Hydro-Québec (compagnie nationale d'électricité), 25~000~CAD
  \vspace{1.3mm} \\

& \faStar~{\heavy Prix MITACS-JSPS pour stage international Canada-Japon} \\
& MITACS -- Société Japonaise pour la Promotion de la Science, 550~000~JPY
  \vspace{1.3mm} \\

& \faStar~{\heavy Bourse de recherche de 3\textsuperscript{e} cycle} \\
& Fonds de Recherche du Québec -- Nature et Technologies, 13~333~CAD
  \vspace{1.3mm} \\

& {\heavy Prix du meilleur poster étudiant} \\
& Frontiers in Plant Reproduction Biology, Conf. \emph{Plant Reproduction 2016}, 300~USD
  \vspace{1.3mm} \\

& {\heavy Bourse de voyage Jacques-Rousseau} \\
& Institut de Recherche en Biologie Végétale, Université de Montréal, 1~500~CAD
  \vspace{1.3mm} \\

& {\heavy Subvention de voyage PARSECS} \\
& FAÉCUM, Université de Montréal, 400~CAD \\

\multicolumn{2}{c}{} \\

2015

& \faStar~{\heavy Bourse d'excellence Catherine-Fradette en sciences biologiques et neurologie} \\
& Faculté des Études Supérieures et Postdoctorales, Univ. de Montréal, 5~000~CAD
  \vspace{1.3mm} \\

& {\heavy Bourse du Fonds de Bourses en Sciences Biologiques (FBSB), niveau doctorat} \\
& Université de Montréal, 1~500~CAD
  \vspace{1.3mm} \\

& {\heavy Prix du Président pour la meilleure présentation orale étudiante} \\
& Société Canadienne de Biologie Végétale (SCBV), Conf. \emph{Botany 2015}, 500~CAD
  \vspace{1.3mm} \\

& {\heavy Prix du meilleur poster étudiant} \\
& Calcul Canada, Symposium de calcul informatique de pointe HPCS 2015, 500~CAD
  \vspace{1.3mm} \\

& {\heavy Bourse de voyage G.-H. Duff} \\
& Société Canadienne de Biologie Végétale (SCBV), 340~CAD \\

\end{tabularx}

\section*{Prix et bourses \small(suite)
          \hfill {\mdseries\faStar}~Bourse ou prix important}

\begin{tabularx}{\textwidth}{@{}r|X@{}}

2015

& {\heavy Bourse de voyage Jacques-Rousseau} \\
& Institut de Recherche en Biologie Végétale, Université de Montréal, 775~CAD
  \vspace{1.3mm} \\

& \faStar~{\heavy Bourse au mérite de la Faculté des Études Supérieures et Postdoctorales} \\
& Université de Montréal, 3~000~CAD\\

\multicolumn{2}{c}{} \\

2014

& {\heavy Bourse Pehr-Kalm} \\
& Jardin botanique de Montréal, 2~000~CAD
  \vspace{1.3mm} \\

& \faStar~{\heavy Bourse pour stagiaires internationaux} \\
& Fonds de Recherche du Québec en Nature et Technologies -- Centre SÈVE, 3~815~CAD \\

& {\heavy Bourse de voyage Jacques-Rousseau} \\
& Institut de Recherche en Biologie Végétale, Université de Montréal, 1~769~CAD\\

\multicolumn{2}{c}{} \\

2013

& \faStar~{\heavy Bourse d'excellence Marie-Victorin} \\
& Institut de Recherche en Biologie Végétale, Université de Montréal, 3~000~CAD
  \vspace{1.3mm} \\

& {\heavy Prix de la meilleure présentation orale} \\
& Journées du Centre SÈVE, 300~CAD
  \vspace{1.3mm} \\

& {\heavy Bourse de voyage Jacques-Rousseau} \\
& Institut de Recherche en Biologie Végétale, Université de Montréal, 850~CAD
  \vspace{1.3mm} \\

& {\heavy Prix de la meilleure présentation orale} \\
& Symposium de biologie de l'Université de Montréal, 100~CAD\\

\multicolumn{2}{c}{} \\

2012

& {\heavy Bourse du Fonds de Bourses en Sciences Biologiques (FBSB), niveau maîtrise} \\
& Université de Montréal, 1~200~CAD
  \vspace{1.3mm} \\

& \faStar~{\heavy Bourse de passage accéléré maîtrise-doctorat} \\
& Faculté des Études Supérieures et Postdoctorales, Univ. de Montréal, 14~000~CAD \\

\multicolumn{2}{c}{} \\

2011

& {\heavy Bourse de voyage pour échange au Québec (Complément mobilité CROUS)} \\
& Ministère français de l'Enseignement Supérieur et de la Recherche, 1~600~EUR
  \vspace{1.3mm} \\

& {\heavy Bourse d'excellence PIL pour échange au Québec} \\
& Université Pierre et Marie Curie (Paris VI), 1~500~EUR
  \vspace{1.3mm} \\

& {\heavy Bourse de voyage AMIÉ pour échange au Québec} \\
& Conseil régional d'Île-de-France (autorités régionales en France), 2~800~EUR
  \vspace{1.3mm} \\

& {\heavy Bourse de voyage Campus'Trotter pour échange au Québec} \\
& Conseil départemental du Morbihan (autorités locales en France), 700~EUR\\

\multicolumn{2}{c}{} \\

2010

& {\heavy Meilleur étudiant aux examens de licence en biologie} \\
& Université Pierre et Marie Curie (Paris VI), semestre S4 \\

\multicolumn{2}{c}{} \\

2008

& \faStar~{\heavy Bourse au mérite CROUS} \\
& Ministère français de l'Enseignement Supérieur et de la Recherche, 5~400~EUR \\

\end{tabularx}

\newpage

% COMMITMENTS
\section{Engagements}

\begin{tabularx}{\textwidth}{@{}r|X@{}}

{\heavy Sociétés}

 & {\heavy American Society of Plant Biologists (ASPB),} {\bfseries depuis 2016}
   \vspace{2mm} \\

 & {\heavy Société Canadienne de Biologie Végétale (CSPB-SCBV),} {\bfseries depuis 2014}
   \vspace{2mm} \\

 & {\heavy International Association of Sexual Plant} \\
 & {\heavy Reproduction Research (IASPRR),} {\bfseries depuis 2015}
   \vspace{2mm} \\

 & {\heavy Association des Biologistes du Québec (ABQ),} {\bfseries 2013--2018}
   \vspace{2mm} \\

 & {\heavy Société Botanique de France (SBF),} {\bfseries 2010--2011} \\

\multicolumn{2}{c}{} \\

{\heavy Associations}
  & {\heavy Association naturaliste \emph{Timarcha},} {\bfseries 2010--2011} \\
{\heavy étudiantes}
  & Université Pierre et Marie Curie (UPMC), Paris, France
    \vspace{2mm} \\

  & {\heavy Comité \emph{Éco-école}} {\bfseries d’actions pour l’environnement, 2006--2008} \\
  & Lycée Saint-Sauveur, Redon, France \\

\multicolumn{2}{c}{} \\

{\heavy Bénévolat}

 & {\heavy Enseignant de français pour des nouveaux arrivants au Canada,} {\bfseries 2015--2016} \\
 & Centre communautaire \emph{La Maison de l’Amitié}, Montréal, QC, Canada \\
 & \textbullet{}~Cours hebdomadaires de 3 h avec 10 à 20 étudiants
   \vspace{2mm} \\

 & {\heavy Contributeur à plusieurs projets en ligne:} \\
 & \textbullet{}~Rédacteur et traducteur pour \emph{Wikipedia}
   (articles de biologie), depuis 2008 \\
 & \textbullet{}~Cartographe bénévole pour \emph{OpenStreetMap},
   since 2015 \\
 & \textbullet{}~Contributeur au projet «~\emph{Les Herbonautes}~» visant à numériser l’herbier du Muséum National d’Histoire Naturelle de Paris, 2015 \\

\end{tabularx}

\vspace{6mm}


% OTHER SKILLS
\section{Autres compétences}
\begin{tabularx}{\textwidth}{@{}r|X@{}}

{\heavy Langues}
& \textbf{Français,} langue maternelle \\
& \textbf{Anglais,} courant \\
& \textbf{Espagnol,} courant \\
& \textbf{Italien,} intermédiaire \\
& \textbf{Espéranto et Japonais,} débutant \\

\multicolumn{2}{c}{} \\

{\heavy Informatique}

& \textbf{Programmation:} Python et R. Bases en C et Perl.
  \vspace{2mm} \\

& \textbf{Internet:} HTML/CSS, Jekyll.
  \vspace{2mm} \\

& \textbf{Systèmes d’exploitation:} Linux (\emph{Ubuntu}, \emph{Fedora},
  \emph{CentOS}), Mac OS X, Windows.
  \vspace{2mm} \\

& \textbf{Bioinformatique:} assembleurs (\emph{Trinity}, \emph{CLC}, etc.);
  aligneurs (\emph{Bowtie}, \emph{TopHat}, etc.);
  outils de recherche et d’alignement de séquences (\emph{BLAST}, etc.);
  annotateurs (\emph{BLAST2GO}, \emph{PFAMscan}, \emph{SignalP}, etc.)
  \vspace{2mm} \\

& \textbf{Bureautique:} \LaTeX, \emph{LibreOffice}/\emph{OpenOffice},
  \emph{Microsoft Office}
  \vspace{2mm} \\

& \textbf{Traitement d’images:} \emph{GIMP}, \emph{Inkscape}, \emph{ImageJ},
  \emph{Adobe Photoshop}, \emph{Cytoscape} ; \emph{AxioVision} (logiciel de commande des microscopes \emph{Zeiss}) \\

\end{tabularx}

\end{document}
