\documentclass[letterpaper,12pt]{article}

% MARGINS
\usepackage[
    top=1.75cm,
    bottom=1.75cm,
    outer=2cm,
    inner=2cm,
    heightrounded,
    marginparwidth=3cm,
    marginparsep=0.5cm]{geometry}


% LANGUAGE
\usepackage{polyglossia}
\setdefaultlanguage{spanish}


% FONTS
\usepackage{fontawesome}
\usepackage{fontspec}
\defaultfontfeatures{Mapping=tex-text}
\setmainfont{Lato}
\newfontfamily{\light}
  [Ligatures=TeX, UprightFont={* Light}, ItalicFont={* Light Italic},
   BoldFont={* Medium}, BoldItalicFont={* Medium Italic}]{Lato}
\newfontfamily{\heavy}
  [Ligatures=TeX, UprightFont={* Heavy}, ItalicFont={* Heavy Italic},
   BoldFont={* Heavy}, BoldItalicFont={* Heavy Italic}]{Lato}


% PAGE LAYOUT
\usepackage{parskip}
\usepackage{titlesec}
\titleformat{\section}{\large\heavy\raggedright}{}{0em}{}[\titlerule]
\titlespacing{\section}{0pt}{3pt}{3pt}


% COLOURS
\usepackage{xcolor}
\definecolor{linkcolour}{rgb}{0,0.2,0.6}


% HYPERLINKS
\usepackage{hyperref}
\hypersetup{
    colorlinks,
    breaklinks,
    urlcolor=linkcolour,
    linkcolor=linkcolour}

% TABLES
\usepackage{multirow}
\usepackage{tabularx}
\newcolumntype{R}{>{\raggedleft\arraybackslash}X}

\usepackage{fancyhdr}
\usepackage{lastpage}
\lhead{}\chead{}\rhead{}\lfoot{}\cfoot{}
\rfoot{\bfseries\small Página \thepage{} de \pageref*{LastPage}}
\renewcommand{\headrulewidth}{0pt}
\renewcommand{\footrulewidth}{0pt}
\footskip=5mm


% DOCUMENT
\begin{document}
\pagestyle{fancy}


% HEADER
\begin{tabularx}{\textwidth}{@{}llll@{}}
  \multicolumn{4}{@{}l@{}}{{\Large\heavy Valentin Joly,} {\Large Ph.D.}}\vspace{1mm} \\
  \multicolumn{4}{@{}l@{}}{\large\bfseries Investigador postdoctoral en biología molecular y bioinformática}\vspace{4mm} \\

  \vspace{0.75mm}
    \faMapMarker~          Universidad Yale, EE UU
  & \faEnvelopeSquare~     \href{mailto:valentin.joly@yale.edu}{valentin.joly@yale.edu}
  & \faExternalLinkSquare~ \href{http://vjoly.net/en/index.html}{http://vjoly.net}
  & \faLinkedinSquare~     \href{https://www.linkedin.com/in/valentinjoly}{valentinjoly} \\

  \vspace{0.75mm}
    \faFlag~               Francés y Canadiense
  & \faPhoneSquare~        +1 (475) 209-6054
  & \faSkype~              valentin.joly
  & \faGithub~             \href{https://github.com/valentinjoly}{valentinjoly} \\

\end{tabularx}

\vspace{4mm}

{\light
Durante mi doctorado en el \href{https://www.irbv.umontreal.ca/chercheurs/daniel-philippe-matton?lang=en}{Matton Lab} de la Universidad de Montreal, me dediqué a la expresión génica ovárica y a la comunicación polen-pistilo en papas silvestres, utilizando la biología molecular y la bioinformática. En 2019, me uní al \href{https://jacob-lab.yale.edu/}{\textbf{Jacob Lab}} de la Universidad de Yale como posdoc, con un nuevo proyecto destinado a revelar funciones ocultas de la heterocromatina de \emph{Arabidopsis} con CRISPR/Cas9. Más información en \href{http://vjoly.net/en/index.html}{vjoly.net}.}
\vspace{5mm}

% EDUCATION
\section{Formación académica}

\begin{tabularx}{\textwidth}{@{}l|X@{}}

  {\heavy Ph.D.}
  & {\heavy Doctorado en ciencias biológicas,} {\bfseries 2019}
    ~~~\small{(“mention Exceptionnel”, $\approx$ \emph{summa cum laude})} \\

  {\heavy M.Sc.}
  & {\heavy Maestría en ciencias biológicas,} {\bfseries 2012}
    ~~~\small{(transición acelerada al doctorado en enero 2013)} \vspace{0.5mm} \\
  & \hspace{1.5mm} Universidad de Montreal, \emph{Montreal, QC, Canadá} \\
  & \hspace{1.5mm} {\small \textbf{Director:} Prof. Daniel P. Matton, Instituto de Investigación en Biología Vegetal (IRBV)} \\
  & \hspace{1.5mm} {\small \textbf{Tesis:} Exploración bioinformática de las interacciones polen-pistilo en \emph{Solanum chacoense}.} \\

  \multicolumn{2}{c}{} \\

  {\heavy B.Sc.}
  & {\heavy Licenciatura en biología, programa internacional,} {\bfseries 2011} \vspace{0.5mm} \\
  & \hspace{1.5mm} Universidad Pierre y Marie Curie (UPMC),
    \emph{París, Francia}: años 1 y 2 \\
  & \hspace{1.5mm} Universidad de Montreal (UdeM),
    \emph{Montreal, QC, Canadá}: año 3 \\

\end{tabularx}

\vspace{6mm}


% RESEARCH EXPERIENCE
\section[Experiencia científica]{Experiencia científica
         \hfill \small{{\mdseries\faFlask}~Biología molecular~~~{\mdseries\faCode}~Bioinformática}}

\begin{tabularx}{\textwidth}{@{}r|X@{}}

{\heavy EE UU}
& {\heavy Investigador postdoctoral becario,} {\bfseries 2019--\emph{present}} \\
& {\em Prof. Yannick Jacob, MCDB, Universidad Yale}
  \vspace{0.5mm} \\
& \small \hspace{1.5mm} \faFlask~Edición del genoma con CRISPR/Cas9.\\
& \small \hspace{1.5mm} \faCode~Transcriptómica y metilómica de alto rendimiento. \\

\multicolumn{2}{c}{} \\

{\heavy Canadá}
& {\heavy Proyecto de doctorado,} {\bfseries desde 2013} \\
& {\em Prof. Daniel P. Matton, IRBV, Universidad de Montreal}
  \vspace{0.5mm} \\
& \small \hspace{1.5mm} \faFlask~Manipulación de ADN y ARN. Clonación. Expresión y purificación de proteínas. \\
& \small \hspace{1.5mm} \faFlask~Cultivo de células vegetales. Ensayos de atracción de tubos polínicos. Microfluídica. \\
& \small \hspace{1.5mm} \faFlask~Microscopía: epifluorescencia, confocal, SEM, TEM. \\
& \small \hspace{1.5mm} \faCode~Programación Python y R. Desarrollo de la herramienta de detección de secuencias KAPPA. \\
& \small \hspace{1.5mm} \faCode~Transcriptómica: Ensambles RNA-seq. Análisis de microarreglos. DGE. Anotación. \\
& \small \hspace{1.5mm} \faCode~Proteómica: Análisis de datos LC-MS. Secretómica. Cuantificación \emph{label-free} de proteínas. \\

\multicolumn{2}{c}{} \\

{\heavy Suecia}
& {\heavy Colaboración internacional,} {\bfseries 2016--2018} \\
& {\em Dr. Johan Edqvist, Universidad de Linköping}
  \vspace{0.5mm} \\
& \small \hspace{1.5mm} \faFlask~Expresión y purificación de proteínas en \emph{Pichia pastoris}. \\
& \small \hspace{1.5mm} \faCode~Desarrollo de una base de datos y de una herramienta de predicción de nsLTPs vegetales. \\

\end{tabularx}

\newpage

\section*{Experiencia científica \small{(continuado)}
          \hfill \small{{\mdseries\faFlask}~Biología molecular~~~{\mdseries\faCode}~Bioinformática}}

\begin{tabularx}{\textwidth}{@{}r|X@{}}

{\heavy Japón}
& {\heavy Programa de verano JSPS,} {\bfseries junio--agosto 2016} \\
& {\em Prof. Tetsuya Higashiyama, ITbM, Universidad de Nagoya}
  \vspace{0.5mm} \\
& \small \hspace{1.5mm} \faFlask~Desarrollo de dispositivos microfluídicos para ensayos de atracción del tubo polínico. \\
& \small \hspace{1.5mm} \faFlask~Introducción a la microscopía de excitación de dos fotones. \\

\multicolumn{2}{c}{} \\

{\heavy EE UU}
& {\heavy Pasantía de investigación internacional,} {\bfseries abril--mayo 2014} \\
& {\em Prof. Willie J. Swanson, Universidad de Washington}
  \vspace{0.5mm} \\
& \small \hspace{1.5mm} \faCode~\emph{Variant calling} (GATK). \\
& \small \hspace{1.5mm} \faCode~Análisis de evolución molecular y de selección positivas (codeml). \\

\multicolumn{2}{c}{} \\

{\heavy Argentina}
& {\heavy Botanical transect,} {\bfseries abril–mayo 2012} \\
& {\em Colaboración con el Dr. Franco E. Chiarini, Universidad Nacional de Córdoba}
  \vspace{0.5mm} \\
& \small \hspace{1.5mm} \faFlask~Colecta de individuos de papas silvestres en la cordillera de los Andes. \\

\multicolumn{2}{c}{} \\

{\heavy Canadá}
& {\heavy Pasantía de investigación,} {\bfseries enero--agosto 2011} \\
& {\em Prof. Daniel P. Matton, Universidad de Montreal}
  \vspace{0.5mm} \\
& \small \hspace{1.5mm} \faFlask~Clonación molecular. Biolística. Epifluorescencia y microscopía confocal. \\

\multicolumn{2}{c}{} \\

{\heavy Francia}
& {\heavy Pasantía de investigación,} {\bfseries junio--julio 2010} \\
& {\em Prof. Christophe Bailly, CNRS/Universidad Pierre y Marie Curie, París}
  \vspace{0.5mm} \\
& \small \hspace{1.5mm} \faFlask~Biología de la dormancia y de la germinación de semillas. \vspace{2.5mm} \\
& {\heavy Pasantía de introducción a la investigación,} {\bfseries enero 2009} \\
& {\em Prof. Chris Bowler, CNRS/École Normale Supérieure, París}
  \vspace{0.5mm} \\
& \small \hspace{1.5mm} \faFlask~Electroforesis de proteínas. Imunoprecipitación. Western Blot. \\

\end{tabularx}

\vspace{6mm}


% EXTRA TRAINING
\section{Formación complementaria}

\begin{tabularx}{\textwidth}{@{}r|X@{}}

\heavy{Bioinformática}
& {\heavy Especialización en bioinformática en línea,} {\bfseries 2016--2018} \\
& \em Universidad de California San Diego, en Coursera \vspace{0.5mm} \\

& {\small \textbf{1.}~{\em Encontrar mensajes escondidos en el ADN.} \textbf{2.}~{\em Secuenciación del genoma.} \textbf{3.}~{\em Comparar genes, proteínas y genomas.} \textbf{4.}~{\em Evolución molecular.} \textbf{5.}~{\em Ciencia de los datos genómicos y \emph{clustering}.} \textbf{6.}~{\em Encontrar mutaciones en el ADN y las protínas.} \textbf{7.}~{\em Proyecto final: \emph{Big data} en biología.}} \\
& {\small{\bfseries Certificado global de la especialización:}~\href{https://www.coursera.org/account/accomplishments/specialization/H528Q2K9KYB6}{H528Q2K9KYB6}} \\

\multicolumn{2}{c}{} \\

\heavy{Python/R}
& {\heavy Cursos de bioinformática en línea,} {\bfseries 2016} \\
& \em Universidad Johns Hopkins, en Coursera \vspace{0.5mm} \\

& \small •~\emph{Python para la ciencia de datos genómicos}. \textbf{Certificado:}~\href{https://www.coursera.org/account/accomplishments/verify/XHKWDB4XD7}{XHKWDB4XD7} \\

& \small •~\emph{Introducción a las tecnologías genómicas}. \textbf{Certificado:}~\href{https://www.coursera.org/account/accomplishments/verify/U88T89XKR2}{U88T89XKR2} \\

& \small •~\emph{Programación con R}. \textbf{Certificado:}~\href{https://www.coursera.org/account/accomplishments/verify/X8NKEQAUU4}{X8NKEQAUU4} \\

\multicolumn{2}{c}{} \\

{\heavy Anotación}
& {\heavy Seminario internacional sobre la anotación funcional de proteínas,} {\bfseries 2012} \\
{\heavy de secuencias}
& \em BLAST2GO, Universidad de California Davis \\

\end{tabularx}

\vspace{6mm}

% PUBLICATIONS
\section[Publicaciones]{Publicaciones \hfill \small{*Co-autores}}

\begin{tabularx}{\textwidth}{@{}r|X@{}}


2019
& \textbf{Joly V*}, Tebbji F*, Nantel A y Matton DP.
  Pollination type recognition from a distance by the ovary is revealed by a
  global transcriptomic analysis.
  \emph{Plants}, 2019, 8(6), 185.
  DOI: \href{http://doi.org/10.3390/plants8060185}{10.3390/plants8060185}
  \vspace{3mm}
  \\

& Mazin BD, \textbf{Joly V} y Matton DP. (2019).
  The ScFRK2 mitogen-activated protein kinase kinase kinase (MAP3K) is involved
  in early embryo sac development in \emph{Solanum chacoense}.
  \emph{Plant Signaling \& Behavior}, 14(8), 1620059.
  DOI: \href{http://doi.org/10.1080/15592324.2019.1620059}
  {10.1080/15592324.2019.1620059}
  \\

\multicolumn{2}{c}{} \\

2018
& Salminen TA, Eklund DM, \textbf{Joly V}, Blomqvist K, Matton DP
  y Edqvist J. (2018).
  Deciphering the evolution and development of the cuticle by studying lipid
  transfer proteins in mosses and liverworts.
  \emph{Plants}, 7(1), 6.
  DOI: \href{http://doi.org/10.3390/plants7010006}{10.3390/plants7010006}
  \\

\multicolumn{2}{c}{} \\

2015
& \textbf{Joly V} y Matton DP. (2015).
  KAPPA, a simple algorithm for the discovery and clustering of proteins defined
  by a key amino acid pattern.
  \emph{Bioinformatics}, 31(11), 1716--1723.
  DOI: \href{http://doi.org/10.1093/bioinformatics/btv047}
  {10.1093/bioinformatics/btv047}
  \vspace{3mm}
  \\

& Liu Y*, \textbf{Joly V*}, Dorion S, Rivoal J y Matton DP. (2015).
  The plant ovule secretome: a different view toward pollen-pistil interactions.
  \emph{Journal of Proteome Research}, 14(11):4763--75.
  DOI: \href{http://doi.org/10.1021/acs.jproteome.5b00618}
  {10.1021/acs.jproteome.5b00618}
  \vspace{3mm}
  \\

& Lafleur É*, Kapfer C*, \textbf{Joly V}, Liu Y, Tebbji F, Daigle C,
  Gray-Mitsumune M, Cappadocia M, Nantel A y Matton DP. (2015).
  The ScFRK1 MAPK kinase kinase (MAPKKK) from \emph{Solanum chacoense} is
  involved in embryo sac and pollen development.
  \emph{Journal of Experimental Botany}, 66(7), 1833--1843.
  DOI: \href{http://doi.org/10.1093/jxb/eru524}{10.1093/jxb/eru524}
  \\

\multicolumn{2}{c}{} \\

{\em venideras}
& \textbf{Joly V*}, Liu Y* y Matton DP.
  Transcriptomic profiling of \emph{Solanum chacoense} mature, immature, and
  embryo sac-less ovules.
  {\bfseries\em Sumisión en septiembre de 2019.}
  \\

\end{tabularx}

\vspace{6mm}

\section[Código informático]{Código informático}

\begin{tabularx}{\textwidth}{@{}r|X@{}}

2015
& \textbf{Joly V} y Matton DP. Key Aminoacid Pattern-based Protein Analyzer
  (KAPPA). \\
& \small \hspace{1.5mm} •~Versión 1.1 publicada bajo licencia GPL en
  \href{https://github.com/valentinjoly/kappa-1.1}{GitHub}. \\
& \small \hspace{1.5mm} •~Versión 1.0 publicada bajo licencia GPL en
  \href{https://sourceforge.net/projects/kappa-sequence-search/}{SourceForge}.
  \\

\end{tabularx}

\newpage

% ORAL PRESENTATIONS
\section[Presentaciones orales]{Presentaciones orales
         \small en conferencias científicas \hfill {\mdseries\faStar}~Premio}

\begin{tabularx}{\textwidth}{@{}r|X@{}}

2017
& \faStar~\textbf{Joly V}, Viallet C, Liu Y, Zaro A, Ceriotti F y Matton DP.
  \emph{Deciphering species-specific pollen tube guidance in \emph{Solanum}.}
  CSPB Eastern Regional Meeting, Montreal, QC, Canadá;
  24–25 noviembre 2017.
  \vspace{1.5mm}
  \\

& \textbf{Joly V}, Viallet C, Liu Y y Matton DP.
  \emph{Reproductive cysteine-rich proteins: key players in \emph{Solanum}
  speciation?}
  Plant Biology 2017, Honolulu, HI, EE UU;
  23–28 junio 2017.
  \\

\multicolumn{2}{c}{} \\

2015
& \faStar~\textbf{Joly V} y Matton DP.
  \emph{Plants’ secret words of love: rapid evolution of pollen–pistil
  recognition proteins drives reproductive isolation of wild potatoes.}
  Botany 2015, Edmonton, AB, Canadá;
  26–29 julio 2015.
  \vspace{1.5mm}
  \\

2013
& \faStar~\textbf{Joly V} y Matton DP.
  \emph{Comment éviter les liaisons dangereuses : secrets d’alcôve des pommes
  de terre.}
  Journées du Centre SÈVE, Wendake, QC, Canadá;
  7–8 noviembre 2013.
  \vspace{1.5mm}
  \\

& \faStar~\textbf{Joly V}, Liu Y y Matton DP.
  \emph{Divergence des protéines reproductives et maintien des barrières de
  spéciation chez les pommes de terre sauvages.}
  23\textsuperscript{e} Symposium des Sciences biologiques,
  Universidad de Montreal, Montreal, QC, Canadá;
  21 marzo 2013.
  \\

\end{tabularx}

\vspace{6mm}

\section[Orador invitado]{Presentaciones orales \small como orador invitado}

\begin{tabularx}{\textwidth}{@{}r|X@{}}

2018
& \textbf{Joly V} y Matton DP.
  \emph{Potato sexomics: deciphering species-specific pollen tube guidance in
  wild potatoes with high-throughput sequencing technologies.}
  Dep. de biología molecular, celular y del desarrollo,
  Universidad Yale, New Haven, CT, EE UU;
  22 octubre 2018.
  \\

\multicolumn{2}{c}{} \\

2016
& \textbf{Joly V} y Matton DP.
  \emph{Pollen tube guidance and reproductive isolation in wild potatoes.}
  Dep. de genómica funcional,
  Universidad de Kanazawa, Japón;
  18 agosto 2016.
  \vspace{1.5mm}
  \\

& \textbf{Joly V} y Matton DP.
  \emph{Species-specific pollen tube guidance in wild potatoes.}
  Laboratorio de biología molecular de plantas,
  Universidad de Kioto, Japón;
  12 agosto 2016.
  \vspace{1.5mm}
  \\

& \textbf{Joly V} y Matton DP.
  \emph{Deciphering potatoes’ words of love.}
  Institute for Transformative bio-Molecules (ITbM),
  Universidad de Nagoya, Japón;
  13 julio 2016.
  \\

\multicolumn{2}{c}{} \\

2015
& \textbf{Joly V} y Matton DP.
  \emph{Sex among wild potatoes: ladies wear the pants.}
  Centro de Genómica Estructural y Funcional,
  Universidad Concordia, Montreal, QC, Canadá;
  16 julio 2015.
  \\

\multicolumn{2}{c}{} \\

2014
& \textbf{Joly V} y Matton DP.
  \emph{Cell-cell communication between gametophytes and reproductive isolation
  in wild potatoes.}
  Dep. de Ciencias genómicas, Universidad de Washington, Seattle, WA, EE UU;
  24 abril 2014.
  \\

\multicolumn{2}{c}{} \\

2013
& \textbf{Joly V} y Matton DP.
  \emph{Species-specificity of pollen-pistil interactions in wild potatoes.}
  Instituto de Genética, Academia de Ciencias de China, Pekín, China;
  24 octubre 2013.
  \\

\end{tabularx}


% POSTER PRESENTATIONS
\section[Presentaciones con póster]{Presentaciones con póster
         \small en conferencias científicas \hfill {\mdseries\faStar}~Premio}

\begin{tabularx}{\textwidth}{@{}r|X@{}}

2018
& \textbf{Joly V} y Matton DP.
  \emph{Long-distance relationships: how the ovary perceives different
  pollination types at a distance.}
  Plant Biology 2018, Montreal, QC, Canadá;
  14–18 julio 2018.
  \\

\multicolumn{2}{c}{} \\

2016
& \faStar~\textbf{Joly V}, Liu Y, Dorion S, Rivoal J y Matton DP.
  \emph{Ovule secretomics reveal the importance of post-transcriptional
  regulation of reproductive proteins.}
  Plant Reproduction 2016, Tucson, AZ, EE UU;
  18–23 marzo 2016.
  \vspace{1.5mm}
  \\

& \faStar~\textbf{Joly V} y Matton DP.
  \emph{KAPPA: exploring -omics data to detect and cluster cysteine-rich
  proteins.}
  [misma conferencia]
  \\

\multicolumn{2}{c}{} \\

2015
& \faStar~\textbf{Joly V} y Matton DP.
  \emph{KAPPA: meeting the challenge of proteome-wide detection and clustering
  of cysteine-rich proteins.}
  High Performance Computing Symposium HPCS 2015, Montreal, QC, Canadá;
  17–19 junio 2015.
  \\

\multicolumn{2}{c}{} \\

2013
& \textbf{Joly V}, Liu Y y Matton DP.
  \emph{Interspecific divergence of reproductive proteins: the keystone of
  species-specific fertilization in wild potatoes?}
  10th Solanaceae Conference (SOL 2013), Pekín, China;
  13–17 octubre 2013.
  \vspace{1.5mm}
  \\

& \textbf{Joly V} y Matton DP.
  \emph{Speciation genes in pollen-pistil interactions.}
  9th Canadian Plant Genomics Workshop, Halifax, NS, Canadá;
  12–15 agosto 2013.
  \\

\end{tabularx}

\vspace{6mm}

% OTHER PRESENTATIONS
\section[Otras presentaciones]{Otras presentaciones \hfill \small{*Presentador}}

\begin{tabularx}{\textwidth}{@{}r|X@{}}

2019
& Mazin BD*, Daigle C, \textbf{Joly V} and Matton DP.
  \emph{The ScFRK2 and ScFRK3 MAP Kinase Kinase Kinase are involved in ovule
  development in \emph{Solanum chacoense}.}
  Plant Biology 2019, San Jose, CA, Canada;
  Aug. 3--7, 2019.
  \\

\multicolumn{2}{c}{} \\

2018
& \textbf{Joly V} y Matton DP*.
  \emph{Pre-zygotic barriers in inter-specific crosses: a leading role for small
  cysteine-rich protein attractant in wild potatoes species ?}
  Plant Biology 2018, Montreal, QC, Canadá;
  14–18 julio 2018.
  \\

\multicolumn{2}{c}{} \\

2017
& \textbf{Joly V} y Matton DP*.
  \emph{Pollination type recognition from a distance by the ovary is revealed
  by a global transcriptomic analysis.}
  5th International Symposium on Plant Signaling and Behavior, Matsue, Japón;
  26 junio – 1 julio 2017.
  \\

\multicolumn{2}{c}{} \\

2013
& Liu Y*, Bai F, \textbf{Joly V} y Matton DP.
  \emph{Identification of female gametophyte-specific CRPs and isolation of
  pollen tube guidance attractant(s) in solanaceous species.}
  Journées du Centre SÈVE, Wendake, QC, Canadá;
  7–8 noviembre 2013.
  \vspace{1.5mm}
  \\

& Tebbji F, \textbf{Joly V} y Matton DP*. \emph{Pollination type recognition
  from a distance by the ovary is revealed by a global transcriptomic analysis.}
  10th Solanaceae Conference (SOL 2013), Pekín, China;
  13–17 octubre 2013.
  \vspace{1.5mm}
  \\

& Liu Y*, \textbf{Joly V} y Matton DP.
  \emph{Isolation and characterization of the pollen tube attractant from}
  Solanum chacoense. [misma conferencia]
  \\

\multicolumn{2}{c}{} \\

2011
& Daigle C*, \textbf{Joly V} y Matton DP.
  \emph{Discovering new MAPK signalling cascades involved in plant reproduction
  using co-expression analyses and deep transcriptomic sequencing of ovule
  and pollen tubes.}
  7th Canadian Plant Genomics Workshop, Niagara Falls, ON, Canadá;
  22–25 agosto 2011.
  \\

\end{tabularx}


\vspace{6mm}

\section[Divulgación científica]{Divulgación científica}

\begin{tabularx}{\textwidth}{@{}r|X@{}}

2016
& \textbf{Joly V}. {\em Le sexe des plantes avec Valentin Joly.} Entrevista de
  radio para el programa de divulgación científica
  \href{http://ici.radio-canada.ca/emissions/les_annees_lumiere/2009-2010/chronique.asp?idChronique=404672}{\emph{Les années lumière}}
  en Radio-Canada. Emitido el 24 de abril de 2016. \\

\multicolumn{2}{c}{} \\

2014
& \textbf{Joly V}. {\em Les mots d’amour des plantes à fleurs.} Artículo escrito
  para \emph{L'ARN messager}, el periódico en línea de los estudiantes de
  biología de la Universidad de Montreal. Publicado el 19 de diciembre de 2014.
  \\

\end{tabularx}


\newpage

% TEACHING
\section{Enseñanza}

\begin{tabularx}{\textwidth}{@{}r|X@{}}

{\heavy Fisiología}
& {\heavy Jefe de asistentes de enseñanza,} {\bfseries 2013--2018} \\
{\heavy vegetal}
& {\heavy Asistente de enseñanza,} {\bfseries 2011--2012} \\
& {\em Trabajos prácticos de fisiología vegetal, Prof.~Jean Rivoal, Universidad de Montreal}
  \vspace{1mm} \\
& •~140 horas por sesión, ~70 estudiantes \\
& •~Sesiones semanales con un curso teorético (0:45) y un trabajo práctico (2:30) \\
& •~Supervisión de 1--2 asistentes de enseñanza \\

\multicolumn{2}{c}{} \\

\heavy{Biología}
& \heavy{Asistente de enseñanza,} {\bfseries 2014--2016} \\
\heavy{molecular}
& {\em Trabajos prácticos de biología molecular, Prof.~D. P. Matton, Universidad de Montreal}
  \vspace{1mm} \\
& •~110 horas per sesión, 10-20 estudiantes \\
\end{tabularx}

\vspace{6mm}

\section{Supervisión de pasantes}

\begin{tabularx}{\textwidth}{@{}r|llll@{}}
{\heavy Posgrado}
 & \multicolumn{4}{X}{\small\em Los siguientes estudiantes latinoamericanos
 fueron recibidos en el laboratorio de mi profesor como parte del Programa de
 Futuros Líderes en las Américas (PFLA-ELAP) organizado por el Gobierno de
 Canadá. Supervisé su trabajo durante de 5 a 6 meses, con proyectos relacionados
 con mi proyecto de doctorado. \vspace{2mm}} \\
 & \textbf{• Kelly Rodrigues} & 2018-19 & Ph.D. & Univ. de São Paulo (Brasil) \\
 & \textbf{• Federico Ceriotti} & 2017-18 & M.Sc. & Univ. Naz. de Cuyo (Argentina) \\
 & \textbf{• Carlos Bravo} & 2016-17 & Ph.D. & Univ. Naz. Autonoma de México (México) \\
 & \textbf{• Laura González} & 2016 & Ph.D. & Univ. Naz. de Córdoba (Argentina) \\
 & \textbf{• Mariana Quiroga} & 2015 & Ph.D. & Univ. Naz. de Córdoba (Argentina) \\

\multicolumn{2}{c}{} \\

{\heavy Grado}
 & \multicolumn{4}{X}{\small\em Supervisé a estos estudiantes durante pasantías
 de 4 a 6 meses que formaban parte de sus carreras de grado. \vspace{2mm}} \\
 & \textbf{• Maude Dorval} & 2018 & B.Sc. & Univ. de Montreal (Canadá) \\
 & & 2017 & DEC & Collège Ahuntsic (Canadá) \\
 & \textbf{• Anna Zaro Sánchez} & 2017 & B.Sc. & Univ. de Barcelona (España) \\
 & \textbf{• Francis Banville} & 2017 & B.Sc. & Univ. de Montreal (Canadá) \\
 & \textbf{• Andréa Davrinche} & 2014 &  B.Sc. & Univ. P. \& M. Curie (Francia) \\
 & \textbf{• Ella Gangbe} &  2013 & B.Sc. & Univ. de Montreal (Canadá) \\
 & \textbf{• Tissicca Hour} &  2012 & B.Sc. & Univ. de Montreal (Canadá) \\
\end{tabularx}

\newpage


% SCHOLARSHIPS AND AWARDS
\section[Premios y becas]{Premios y becas
         \hfill \small{{\mdseries\faStar}~Premio o beca importante}}

\begin{tabularx}{\textwidth}{@{}r|X@{}}

2019--21

& \faStar~\textbf{Beca de investigación postdoctoral (B3X)} \\
& \emph{Fonds de Recherche du Québec -- Nature et Technologies} (FRQNT) \\
& Gobierno de Québec, Canadá, 110~000~CAD \\

\multicolumn{2}{c}{} \\

2018

& \textbf{Beca de viaje Jacques-Rousseau} \\
& Instituto de Investigación en Biología Vegetal, Universidad de Montreal, 800~CAD \\

\multicolumn{2}{c}{} \\

2017

& \faStar~\textbf{Beca de excelencia Hydro-Québec (2ᵒ año)} \\
& Hydro-Québec (compañía nacional de electricidad), 25 000~CAD
  \vspace{1.3mm} \\

& \textbf{Beca de fin de doctorado (BFED)} \\
& Facultad de Estudios de Grado y Posgrado, Universidad de Montreal, 8 400~CAD
  \vspace{1.3mm} \\

& \textbf{Beca de viaje Jacques-Rousseau} \\
& Instituto de Investigación en Biología Vegetal, Universidad de Montreal, 1 500~CAD
  \vspace{1.3mm} \\

& \textbf{Beca de viaje (\emph{Bourse d'appui à la diffusion des résultats de recherche})} \\
& Facultad de Estudios de Grado y Posgrado, Universidad de Montreal, 500~CAD
  \vspace{1.3mm} \\

& \textbf{Mención honorífica para una presentación oral estudiante} \\
& CSPB Eastern Regional Meeting \\

\multicolumn{2}{c}{} \\

2016

& \faStar~\textbf{Beca de excelencia Hydro-Québec} \\
& Hydro-Québec (compañía nacional de electricidad), 25 000~CAD
  \vspace{1.3mm} \\

& \faStar~\textbf{Beca de doctorado} \\
& Gobierno de Québec (FRQNT), 13 333~CAD
  \vspace{1.3mm} \\

& \faStar~\textbf{Premio MITACS Globalink – Programa de verano de la JSPS} \\
& MITACS / Japanese Society for the Promotion of Science, 534 000~JPY
  \vspace{1.3mm} \\

& \textbf{Premio del mejor poster estudiante} \\
& Frontiers in Plant Reproduction Biology, conf. \emph{Plant Reproduction 2016}, 300~USD
  \vspace{1.3mm} \\

& \textbf{Beca de viaje Jacques-Rousseau} \\
& Instituto de Investigación en Biología Vegetal, Universidad de Montreal, 1 500~CAD
  \vspace{1.3mm} \\

& \textbf{Subvención de viaje PARSECS} \\
& FAÉCUM, Universidad de Montreal, 400~CAD \\

\multicolumn{2}{c}{} \\

2015

& \faStar~\textbf{Beca de excelencia Catherine-Frédette en ciencias biológicas y neurología} \\
& Facultad de Estudios de Grado y Posgrado, Universidad de Montreal, 5 000~CAD
  \vspace{1.3mm} \\

& \faStar~\textbf{Beca de doctorado FBSB del Departamento de Ciencias biológicas} \\
& Universidad de Montreal, 1 500~CAD
  \vspace{1.3mm} \\

& \textbf{Premio del presidente para la mejor presentación oral} \\
& Sociedad Canadiense de Biología Vegetal (CSPB-SCBV), 500~CAD
  \vspace{1.3mm} \\

& \textbf{Premio de la mejor presentación oral estudiante} \\
& Compute Canada, High Performance Computing Symposium HPCS 2015, 500~CAD
  \vspace{1.3mm} \\

& \textbf{Beca de viaje G.-H. Duff} \\
& Sociedad Canadiense de Biología Vegetal (CSPB-SCBV), 340~CAD \\

\end{tabularx}

\section*{Premios y becas \small{(continuado)}
          \hfill \small{{\mdseries\faStar}~Premio o beca importante}}

\begin{tabularx}{\textwidth}{@{}r|X@{}}

2015

& \textbf{Beca de viaje Jacques-Rousseau} \\
& Instituto de Investigación en Biología Vegetal, Universidad de Montreal, 775~CAD
  \vspace{1.3mm} \\

& \faStar~\textbf{Beca de excelencia de la Facultad de Estudios de Grado y Posgrado} \\
& Universidad de Montreal, 3 000~CAD \\

\multicolumn{2}{c}{} \\

2014

& \faStar~\textbf{Beca Pehr-Kalm} \\
& Jardín Botánico de Montreal, 2 000~CAD
  \vspace{1.3mm} \\

& \textbf{Beca de viaje para pasantías internacionales} \\
& Gobierno de Québec (FRQNT) – Centre SÈVE, 3 815~CAD
  \vspace{1.3mm} \\

& \textbf{Beca de viaje Jacques-Rousseau} \\
& Instituto de Investigación en Biología Vegetal, Universidad de Montreal, 1 760~CAD \\

\multicolumn{2}{c}{} \\

2013

& \faStar~\textbf{Beca de excelencia Marie-Victorin} \\
& Instituto de Investigación en Biología Vegetal, Universidad de Montreal, 3 000~CAD
  \vspace{1.3mm} \\

& \textbf{Premio de la mejor presentación oral} \\
& Journées du Centre SÈVE, 300~CAD
  \vspace{1.3mm} \\

& \textbf{Beca de viaje Jacques-Rousseau} \\
& Instituto de Investigación en Biología Vegetal, Universidad de Montreal, 850~CAD
  \vspace{1.3mm} \\

& \textbf{Premio de la mejor presentación oral estudiante} \\
& Symposium of Biological Sciences, Universidad de Montreal, 100~CAD \\

\multicolumn{2}{c}{} \\

2012

& \faStar~\textbf{Beca de maestría FBSB del Departamento de Ciencias biológicas} \\
& Universidad de Montreal, 1 200~CAD
  \vspace{1.3mm} \\

& \faStar~\textbf{Beca de transición maestría–doctorado acelerada} \\
& Facultad de Estudios de Grado y Posgrado, Universidad de Montreal, 14 000~CAD \\

\multicolumn{2}{c}{} \\

2011

& \textbf{Beca de viaje para un intercambio en Canadá} \\
& Ministerio francés de Investigación (CROUS), 1 600~EUR
  \vspace{1.3mm} \\

& \textbf{Beca de excelencia PIL para intercambios en Canadá} \\
& Universidad Pierre y Marie Curie (París VI), 1 500~EUR
  \vspace{1.3mm} \\

& \textbf{Beca de viaje AMIÉ para un intercambio en Canadá} \\
& Autoridad regional de París (\emph{Conseil régional}), 2 800~EUR
  \vspace{1.3mm} \\

& \textbf{Beca de viaje Campus'Trotter para un intercambio en Canadá} \\
& Autoridad local en Francia (\emph{Conseil général}), 700~EUR \\

\multicolumn{2}{c}{} \\

2010

& \textbf{Mejor estudiante en el departamento de biología, exámenes de junio 2010} \\
& Universidad Pierre y Marie Curie (París VI) \\

\multicolumn{2}{c}{} \\

2008

& \faStar~\textbf{Beca de excelencia para estudios de licenciatura} \\
& Gobierno francés (CROUS), 5 400~EUR \\

\end{tabularx}

\newpage

% COMMITMENTS
\section{Servicio}

\begin{tabularx}{\textwidth}{@{}r|X@{}}

{\heavy Sociedades}

 & {\heavy Sociedad Estadounidense de Biología Vegetal (ASPB),} {\bfseries desde 2016}
   \vspace{2mm} \\

 & {\heavy Sociedad Canadiense de Biología Vegetal (SCBV-CSPB),} {\bfseries desde 2014}
   \vspace{2mm} \\

 & {\heavy Asociación Internacional de Investigación} \\
 & {\heavy en Reproducción Sexual de Plantas (IASPRR),} {\bfseries desde 2015}
   \vspace{2mm} \\

 & {\heavy Asociación de Biólogos de Quebec (ABQ),} {\bfseries 2013--2018}
   \vspace{2mm} \\

 & {\heavy Sociedad Botánica de Francia (SBF),} {\bfseries 2010--2011}
   \\

\multicolumn{2}{c}{} \\

{\heavy Asociaciones}
  & {\heavy Asociación de estudiantes naturalistas \emph{Timarcha},} {\bfseries 2010--2011} \\
{\heavy de estudiantes}
  & Universidad Pierre y Marie Curie (UPMC), París, Francia
    \vspace{2mm} \\

  & {\heavy Comité de acciones ambientales \emph{Éco-école},} {\bfseries 2006--2008} \\
  & Lycée Saint-Sauveur ($\approx$ escuela secundaria), Redon, Francia \\

\multicolumn{2}{c}{} \\

{\heavy Voluntariado}

 & {\heavy Profesor voluntario de francés,} {\bfseries 2015--2016} \\
 & Centro comunitario \emph{La Casa de la Amistad}, Montreal, QC, Canadá \\
 & •~Lecciones de 3 horas cada semana con 10-20 estudiantes
   \vspace{2mm} \\

 & {\heavy Colaborador en diversos proyectos en línea:} \\
 & •~Escritor y traductor para \emph{Wikipedia} (artículos relacionados con la biología), desde 2008 \\
 & •~Cartógrafo voluntario para \emph{OpenStreetMap}, desde 2015 \\
 & •~Digitalización de herbarios para el Museo Nacional de Historia Natural de París (Proyecto “\emph{Les Herbonautes}”), 2015 \\

\end{tabularx}

\vspace{6mm}

% OTHER SKILLS
\section{Otras competencias}

\begin{tabularx}{\textwidth}{@{}r|X@{}}

{\heavy Lenguas}
& \textbf{Frencés,} lengua materna \\
& \textbf{Inglés,} avanzado \\
& \textbf{Español,} avanzado \\
& \textbf{Italiano,} intermedio \\
& \textbf{Esperanto y Japonés,} principiante \\

\multicolumn{2}{c}{} \\

{\heavy Informática}
& \textbf{Programación:} Python y R. Bases de C y Perl.
  \vspace{2mm} \\

& \textbf{Web:} HTML/CSS, Jekyll.
  \vspace{2mm} \\

& \textbf{Sistemas operativos:} Linux (\emph{Ubuntu}, \emph{Fedora},
  \emph{CentOS}), Mac OS X, Windows.
  \vspace{2mm} \\

& \textbf{Bioinformática:} programas de ensamble (\emph{Trinity}, \emph{CLC}, etc.);
  alineadores de lécturas (\emph{Bowtie}, \emph{TopHat}, etc.);
  herremientas de búsqueda y alineamiento de secuencias (\emph{BLAST}, etc.);
  anotadores (\emph{BLAST2GO}, \emph{PFAMscan}, \emph{SignalP}, etc.)
  \vspace{2mm} \\

& \textbf{Software de oficina:} \LaTeX, \emph{LibreOffice}/\emph{OpenOffice},
  \emph{Microsoft Office}
  \vspace{2mm} \\

& \textbf{Procesamiento de imágenes:} \emph{GIMP}, \emph{Inkscape}, \emph{ImageJ},
  \emph{Adobe Photoshop}, \emph{Cytoscape} ; \emph{AxioVision} (programa de manejo de los microscopios \emph{Zeiss}) \\

\end{tabularx}

\end{document}
