\documentclass[letterpaper,12pt]{article}

% MARGINS
\usepackage[
    top=1.75cm,
    bottom=1.75cm,
    outer=2cm,
    inner=2cm,
    heightrounded,
    marginparwidth=3cm,
    marginparsep=0.5cm]{geometry}


% LANGUAGE
\usepackage{polyglossia}
\setdefaultlanguage{esperanto}


% FONTS
\usepackage{fontawesome}
\usepackage{fontspec}
\defaultfontfeatures{Mapping=tex-text}
\setmainfont{Lato}
\newfontfamily{\light}
  [Ligatures=TeX, UprightFont={* Light}, ItalicFont={* Light Italic},
   BoldFont={* Medium}, BoldItalicFont={* Medium Italic}]{Lato}
\newfontfamily{\heavy}
  [Ligatures=TeX, UprightFont={* Heavy}, ItalicFont={* Heavy Italic},
   BoldFont={* Heavy}, BoldItalicFont={* Heavy Italic}]{Lato}


% PAGE LAYOUT
\usepackage{parskip}
\usepackage{titlesec}
\titleformat{\section}{\large\heavy\raggedright}{}{0em}{}[\titlerule]
\titlespacing{\section}{0pt}{3pt}{3pt}


% COLOURS
\usepackage{xcolor}
\definecolor{linkcolour}{rgb}{0,0.2,0.6}


% HYPERLINKS
\usepackage{hyperref}
\hypersetup{
    colorlinks,
    breaklinks,
    urlcolor=linkcolour,
    linkcolor=linkcolour}

% TABLES
\usepackage{multirow}
\usepackage{tabularx}
\newcolumntype{R}{>{\raggedleft\arraybackslash}X}

\usepackage{fancyhdr}
\usepackage{lastpage}
\lhead{}\chead{}\rhead{}\lfoot{}\cfoot{}
\rfoot{\bfseries\small Paĝo \thepage{} de \pageref*{LastPage}}
\renewcommand{\headrulewidth}{0pt}
\renewcommand{\footrulewidth}{0pt}
\footskip=5mm


% DOCUMENT
\begin{document}
\pagestyle{fancy}


% HEADER
\begin{tabularx}{\textwidth}{@{}lllR@{}}
  \multicolumn{3}{@{}l@{}}{\Large\heavy Valentin Joly}\vspace{1mm} & \multirow{4}{*}{\includegraphics[width=2.6cm]{../valentin.jpg}} \\
  \multicolumn{3}{@{}l@{}}{\large\bfseries Molekula biologo • Biokomputikisto}\vspace{4mm} &  \\
  \vspace{0.75mm} \faPhoneSquare~+1 (438) 495-3215
  & \faEnvelopeSquare~\href{mailto:valentin.joly@gmail.com}{valentin.joly@gmail.com}
  & \faLinkedinSquare~\href{https://www.linkedin.com/in/valentinjoly}{valentinjoly} & \\

  \faSkype~valentin.joly
  & \faExternalLinkSquare~\href{http://vjoly.net/eo/index.html}{http://vjoly.net}
  & \faGithub~\href{https://github.com/valentinjoly}{valentinjoly} & \\
\end{tabularx}

\vspace{4mm}

{\light Doktorigxonto cxe la laboratorio de la P-ro Matton, mi celas malkovri la
molekulajn mekanismojn kiuj kontrolas prezigotajn reproduktajn barojn en sovagxaj
terpomoj. Mi speciale interesigxas pri atrakcio de polen-tuboj, kun duobla
metodo, kiu kombinas biokomputiko (RNA-vicrivelado) kaj molekula biologio
(proteino-esprimo kaj funkciaj analizoj).
\emph{Pli da informoj cxe \href{http://vjoly.net/eo/index.html}{vjoly.net}.}}
\vspace{5mm}




% EDUCATION
\section{Edukado}

\begin{tabularx}{\textwidth}{@{}l|X@{}}

  {\heavy Ph.D.}
  & {\heavy Doktoriĝo pri biologio,} {\bfseries ekde 2013}
    ~~~\small{(tezo prezentota en junio 2019)} \\

  {\heavy M.Sc.}
  & {\heavy Magistro pri biologio,} {\bfseries 2012}
    ~~~\small{(akcelita transiro al doktoriĝo en 2013)} \vspace{0.5mm} \\
  & \hspace{1.5mm} Universitato de Montrealo, \emph{Montrealo, QC, Kanado} \\
  & \hspace{1.5mm} {\small \textbf{Projekto:} Molekula komunikado inter masklaj kaj inaj gametofitoj,} \\
  & \hspace{1.5mm} {\small \phantom{\textbf{Projekto:}} kaj reproduktaj baroj en sovaĝaj terpomoj (\emph{Solanum} sect. \emph{Petota}).} \\

  \multicolumn{2}{c}{} \\

  {\heavy B.Sc.}
  & {\heavy Bakalaŭro pri biologio, internacia programo,} {\bfseries 2011} \vspace{0.5mm} \\
  & \hspace{1.5mm} Universitato Pierre-kaj-Marie-Curie (UPMC),
    \emph{Parizo, Francio}: jaroj 1 kaj 2 \\
  & \hspace{1.5mm} Universitato de Montrealo (UdeM),
    \emph{Montrealo, QC, Kanado}: jaro 3 \\

\end{tabularx}

\vspace{6mm}


% RESEARCH EXPERIENCE
\section[Scienca sperto]{Scienca sperto
         \hfill \small{{\mdseries\faFlask}~Molekula biologio~~~{\mdseries\faCode}~Biokomputiko}}

\begin{tabularx}{\textwidth}{@{}r|X@{}}

{\heavy Kanado}
& {\heavy Doktora esplorprojekto} {\bfseries ekde 2013} \\
& {\em P-ro Daniel P. Matton, IRBV, Universitato de Montrealo}
  \vspace{0.5mm} \\
& \small \hspace{1.5mm} \faFlask~DNA- kaj RNA-teknikoj. Klonado. Esprimo kaj purigo de proteinoj. \\
& \small \hspace{1.5mm} \faFlask~Kulturo de plantaj ĉeloj. Eksperimentoj de aktraksio de polen-tuboj. Mikrofluidiko. \\
& \small \hspace{1.5mm} \faFlask~Mikroskopio: epifluoreska, konfokusa kaj elektrona mikroskopio (SEM, TEM). \\
& \small \hspace{1.5mm} \faCode~Python kaj R programado. Disvolviĝo de la sekvenca serĉilo KAPPA. \\
& \small \hspace{1.5mm} \faCode~Transcriptomiko: RNA-vicrivelado. DNA-mikropuntaro. Genesprimiĝo-analizo. Anotado. \\
& \small \hspace{1.5mm} \faCode~Proteomiko: LC-MS-datuma analizo. Sekretomiko. \\

\multicolumn{2}{c}{} \\

{\heavy Svedio}
& {\heavy Internacia kunlaboro,} {\bfseries 2016--2018} \\
& {\em D-ro Johan Edqvist, Universitato de Linköping}
  \vspace{0.5mm} \\
& \small \hspace{1.5mm} \faFlask~Esprimo kaj purado de proteinoj en \emph{Pichia pastoris}. \\
& \small \hspace{1.5mm} \faCode~Disvolviĝo de nsLTP-prognozilo kaj datumbazo de plantaj nsLTP. \\

\multicolumn{2}{c}{} \\

{\heavy Japanio}
& {\heavy Somera programo de la JSPS,} {\bfseries junio–aŭgusto 2016} \\
& {\em P-ro Tetsuya Higashiyama, ITbM, Universitato de Nagojo}
  \vspace{0.5mm} \\
& \small \hspace{1.5mm} \faFlask~Disvolviĝo de mikrofluidikaj aparatoj por ekzamenoj pri polen-tuboj. \\
& \small \hspace{1.5mm} \faFlask~Enkonduko al la 2-fotona konfokusa mikroskopio. \\

\end{tabularx}

\newpage

\section*{Scienca sperto \small{(daŭrigo)}
          \hfill \small{{\mdseries\faFlask}~Molekula biologio~~~{\mdseries\faCode}~Biokomputiko}}

\begin{tabularx}{\textwidth}{@{}r|X@{}}

{\heavy Usono}
& {\heavy Internacia staĝo de esplorado,} {\bfseries aprilo–majo 2014} \\
& {\em P-ro Willie J. Swanson, Universitato de Vaŝingtonio}
  \vspace{0.5mm} \\
& \small \hspace{1.5mm} \faCode~Analizo de genetikaj variantoj (GATK). \\
& \small \hspace{1.5mm} \faCode~Analizoj de molekula evoluado kaj pozitiva selektado (codeml). \\

\multicolumn{2}{c}{} \\

{\heavy Argentino}
& {\heavy Botanika vojaĝo,} {\bfseries aprilo–majo 2012} \\
& {\em Kunlaborado kun la D-ro Franco E. Chiarini, Nacia Universitato de Kordobo}
  \vspace{0.5mm} \\
& \small \hspace{1.5mm} \faFlask~Kolekto de sovaĝaj terpomoj en la Andoj. \\

\multicolumn{2}{c}{} \\

{\heavy Kanado}
& {\heavy Staĝo de esplorado,} {\bfseries januaro–aŭgusto 2011} \\
& {\em P-ro Daniel P. Matton, Universitato de Montrealo}
  \vspace{0.5mm} \\
& \small \hspace{1.5mm} \faFlask~Molekula klonado. Biolistiko. Epifluoreska kaj konfokusa mikroskopio. \\

\multicolumn{2}{c}{} \\

{\heavy Francio}
& {\heavy Staĝo de esplorado,} {\bfseries junio–julio 2010} \\
& {\em P-ro Christophe Bailly, CNRS/Universitato Pierre-kaj-Marie-Curie, Parizo}
  \vspace{0.5mm} \\
& \small \hspace{1.5mm} \faFlask~Biologio de sema dormado kaj ĝermado. \vspace{2.5mm} \\
& {\heavy Mallongatempa enkonduka staĝo de esplorado,} {\bfseries januaro 2009} \\
& {\em P-ro Chris Bowler, CNRS/Normala Altlernejo de Parizo (ENS), Parizo}
  \vspace{0.5mm} \\
& \small \hspace{1.5mm} \faFlask~Elektroforezo de proteinoj. Imunoprecipitado. “Western Blot”. \\

\end{tabularx}

\vspace{6mm}


% EXTRA TRAINING
\section{Alia trejnado}

\begin{tabularx}{\textwidth}{@{}r|lX@{}}

\heavy{Biokomputiko}
& \multicolumn{2}{l}{{\heavy Enreta trejnado pri biokomputiko,} {\bfseries 2016--2018}} \\
& \multicolumn{2}{l}{\em Universitato de Kalifornio ĉe San-Diego, en Coursera \vspace{0.5mm}} \\

& \small \hspace{1.5mm} {\bfseries 1.} {\em Trovi kaŝitaj mesaĝoj en la DNA.}
& \small Atestilo \href{https://www.coursera.org/account/accomplishments/verify/SPRUS2D6NH}{SPRUS2D6NH} \\

& \small \hspace{1.5mm} {\bfseries 2.} {\em Genara vicrivelado.}
& \small Atestilo \href{https://www.coursera.org/account/accomplishments/verify/73HUUXBY64}{73HUUXBY64} \\

& \small \hspace{1.5mm} {\bfseries 3.} {\em Kompari genojn, proteinojn kaj genarojn.}
& \small Atestilo \href{https://www.coursera.org/account/accomplishments/verify/HY7JCN6UV2}{HY7JCN6UV2} \\

& \small \hspace{1.5mm} {\bfseries 4.} {\em Molekula evoluado.}
& \small Atestilo \href{https://www.coursera.org/account/accomplishments/verify/VYKM2WT4792A}{VYKM2WT4792A} \\

& \small \hspace{1.5mm} {\bfseries 5.} {\em Genara datumscienco kaj aretanalizo.}
& \small Atestilo \href{https://www.coursera.org/account/accomplishments/verify/M6ZPV8VCEH}{M6ZPV8VCEH} \\

& \small \hspace{1.5mm} {\bfseries 6.} {\em Trovi mutaciojn en la DNA kaj la proteinoj.}
& \small Atestilo \href{https://www.coursera.org/account/accomplishments/verify/EVDAXLXV9L}{EVDAXLXV9L} \\

& \small \hspace{1.5mm} {\bfseries 7.} {\em Fina Projekto: Grandaj datenoj en biologio.}
& \small Atestilo \href{https://www.coursera.org/account/accomplishments/verify/56XJX7TMHYVM}{56XJX7TMHYVM} \\

& \small \hspace{1.5mm} {\bfseries Fina atestilo.}
& \small Atestilo \href{https://www.coursera.org/account/accomplishments/specialization/H528Q2K9KYB6}{H528Q2K9KYB6} \\

\multicolumn{2}{c}{} \\

\heavy{Python/R}
& \multicolumn{2}{l}{{\heavy Enretaj kursoj pri biokomputiko,} {\bfseries 2016}} \\
& \multicolumn{2}{l}{\em Johns Hopkins University, on Coursera \vspace{0.5mm}} \\

& \small \hspace{1.5mm} •~\emph{Python por genara datumscienco}.
& \small Atestilo \href{https://www.coursera.org/account/accomplishments/verify/XHKWDB4XD7}{XHKWDB4XD7} \\

& \small \hspace{1.5mm} •~\emph{Enkonduko al genomikaj teknologioj }.
& \small Atestilo \href{https://www.coursera.org/account/accomplishments/verify/U88T89XKR2}{U88T89XKR2} \\

& \small \hspace{1.5mm} •~\emph{Programado R}.
& \small Atestilo \href{https://www.coursera.org/account/accomplishments/verify/X8NKEQAUU4}{X8NKEQAUU4} \\

\multicolumn{2}{c}{} \\

{\heavy Anotado}
& \multicolumn{2}{l}{{\heavy Internacia seminario pri unkcia anotado de proteinoj,} {\bfseries 2012}} \\
{\heavy de proteinoj}
& \multicolumn{2}{l}{\em BLAST2GO, Universitato de Kalifornio ĉe Davis} \\

\end{tabularx}


\vspace{6mm}

% PUBLICATIONS
\section[Publikaĵoj]{Publikaĵoj \hfill \small{*Ko-unuaj aŭtoroj}}

\begin{tabularx}{\textwidth}{@{}r|X@{}}

2018
& Salminen TA, Eklund DM, \textbf{Joly V}, Blomqvist K, Matton DP
  kaj Edqvist J. (2018).
  Deciphering the evolution and development of the cuticle by studying lipid
  transfer proteins in mosses and liverworts.
  \emph{Plants}, 7(1), 6.
  DOI: \href{http://doi.org/10.3390/plants7010006}{10.3390/plants7010006}
  \\

\multicolumn{2}{c}{} \\

2015
& \textbf{Joly V} kaj Matton DP. (2015).
  KAPPA, a simple algorithm for the discovery and clustering of proteins defined
  by a key amino acid pattern.
  \emph{Bioinformatics}, 31(11), 1716--1723.
  DOI: \href{http://doi.org/10.1093/bioinformatics/btv047}
  {10.1093/bioinformatics/btv047}
  \vspace{3mm}
  \\

& Liu Y*, \textbf{Joly V*}, Dorion S, Rivoal J kaj Matton DP. (2015).
  The plant ovule secretome: a different view toward pollen-pistil interactions.
  \emph{Journal of Proteome Research}, 14(11):4763--75.
  DOI: \href{http://doi.org/10.1021/acs.jproteome.5b00618}
  {10.1021/acs.jproteome.5b00618}
  \vspace{3mm}
  \\

& Lafleur É*, Kapfer C*, \textbf{Joly V}, Liu Y, Tebbji F, Daigle C,
  Gray-Mitsumune M, Cappadocia M, Nantel A kaj Matton DP. (2015).
  The ScFRK1 MAPK kinase kinase (MAPKKK) from \emph{Solanum chacoense} is
  involved in embryo sac and pollen development.
  \emph{Journal of Experimental Botany}, 66(7), 1833--1843.
  DOI: \href{http://doi.org/10.1093/jxb/eru524}{10.1093/jxb/eru524}
  \\

\multicolumn{2}{c}{} \\

{\em venontaj}
& \textbf{Joly V}, Tebbji F kaj Matton DP.
  Pollination type recognition from a distance by the ovary is revealed by a
  global transcriptomic analysis.
  {\bfseries\em Submetota en oktobro 2018.}
  \vspace{3mm}
  \\

& \textbf{Joly V}, Liu Y kaj Matton DP.
  Comparative RNA-sequencing reveals female gameotphyte-sac specific transcripts
  in the \emph{frk1} embryo sac-less mutant from \emph{Solanum chacoense}.
  {\bfseries\em Submetota en decembro 2018.}
  \vspace{3mm}
  \\

& \textbf{Joly V} kaj Matton DP.
  A transcriptomic time-course reveals developmentally regulated transcripts
  during ovule genesis and maturation in \emph{Solanum chacoense}.
  {\bfseries\em Submetota en marto 2019.} \\

\end{tabularx}

\vspace{6mm}

\section[Komputila kodo]{Komputila kodo}

\begin{tabularx}{\textwidth}{@{}r|X@{}}

2015
& \textbf{Joly V} kaj Matton DP. Key Aminoacid Pattern-based Protein Analyzer
  (KAPPA). \\
& \small \hspace{1.5mm} •~Versio 1.1 publikita sub permesilo GPL en
  \href{https://github.com/valentinjoly/kappa-1.1}{GitHub}. \\
& \small \hspace{1.5mm} •~Versio 1.0 publikita sub permesilo GPL en
  \href{https://sourceforge.net/projects/kappa-sequence-search/}{SourceForge}.
  \\

\end{tabularx}

\vspace{6mm}

\section[Scienca popularigo]{Scienca popularigo}

\begin{tabularx}{\textwidth}{@{}r|X@{}}

2016
& \textbf{Joly V}. {\em Le sexe des plantes avec Valentin Joly.} Radio-intervjuo
  por la scienca populara programo
  \href{http://ici.radio-canada.ca/emissions/les_annees_lumiere/2009-2010/chronique.asp?idChronique=404672}{\emph{Les années lumière}}
  en Radio-Canada. Disdonita la 24-an de aprilo 2016. \\

\multicolumn{2}{c}{} \\

2014
& \textbf{Joly V}. {\em Les mots d’amour des plantes à fleurs}. Artikolo
  skribita por \emph{L'ARN messager}, la enreta ĵuranlo de la biologiaj
  studentoj de la Universitato de Montrealo. Publikita la 19-a de decembro 2014.
  \\

\end{tabularx}

\newpage

% ORAL PRESENTATIONS
\section[Parolaj prezentoj]{Parolaj prezentoj
         \small en kongresoj kun kolega revizio \hfill {\mdseries\faStar}~Premio}

\begin{tabularx}{\textwidth}{@{}r|X@{}}

2017
& \faStar~\textbf{Joly V}, Viallet C, Liu Y, Zaro A, Ceriotti F kaj Matton DP.
  \emph{Deciphering species-specific pollen tube guidance in \emph{Solanum}.}
  CSPB Eastern Regional Meeting, Montrealo, QC, Kanado;
  24–25 novembro 2017.
  \vspace{1.5mm}
  \\

& \textbf{Joly V}, Viallet C, Liu Y kaj Matton DP.
  \emph{Reproductive cysteine-rich proteins: key players in \emph{Solanum}
  speciation?}
  Plant Biology 2017, Honolulu, HI, Usono;
  23–28 junio 2017.
  \\

\multicolumn{2}{c}{} \\

2015
& \faStar~\textbf{Joly V} kaj Matton DP.
  \emph{Plants’ secret words of love: rapid evolution of pollen–pistil
  recognition proteins drives reproductive isolation of wild potatoes.}
  Botany 2015, Edmontono, AB, Kanado;
  26–29 julio 2015.
  \vspace{1.5mm}
  \\

2013
& \faStar~\textbf{Joly V} kaj Matton DP.
  \emph{Comment éviter les liaisons dangereuses : secrets d’alcôve des pommes
  de terre.}
  Journées du Centre SÈVE, Wendake, QC, Kanado;
  7–8 novembro 2013.
  \vspace{1.5mm}
  \\

& \faStar~\textbf{Joly V}, Liu Y kaj Matton DP.
  \emph{Divergence des protéines reproductives et maintien des barrières de
  spéciation chez les pommes de terre sauvages.}
  23\textsuperscript{e} Symposium des Sciences biologiques,
  Universitato de Montrealo, Montrealo, QC, Kanado;
  21 marto 2013.
  \\

\end{tabularx}

\vspace{6mm}

\section[Invitata parolisto]{Parolaj prezentoj \small kiel invitata parolisto}

\begin{tabularx}{\textwidth}{@{}r|X@{}}

2018
& \textbf{Joly V} kaj Matton DP.
  \emph{Potato sexomics: deciphering species-specific pollen tube guidance in
  wild potatoes with high-throughput sequencing technologies.}
  Dep. de Molekula, Ĉela kaj Disvolviĝa Biologio,
  Universitato Yale, Nov-Haveno, CT, Usono;
  22 oktobro 2018.
  \\

\multicolumn{2}{c}{} \\

2016
& \textbf{Joly V} kaj Matton DP.
  \emph{Pollen tube guidance and reproductive isolation in wild potatoes.}
  Dep. de Funkcia genaro,
  Universitato de Kanazaŭa, Japanio;
  18 aŭgusto 2016.
  \vspace{1.5mm}
  \\

& \textbf{Joly V} kaj Matton DP.
  \emph{Species-specific pollen tube guidance in wild potatoes.}
  Laboratorio de Molekula biologio de plantoj,
  Universitato de Kioto, Japanio;
  12 aŭgusto 2016.
  \vspace{1.5mm}
  \\

& \textbf{Joly V} kaj Matton DP.
  \emph{Deciphering potatoes’ words of love.}
  Institute for Transformative bio-Molecules (ITbM),
  Universitato de Nagojo, Japanio;
  13 julio 2016.
  \\

\multicolumn{2}{c}{} \\

2015
& \textbf{Joly V} kaj Matton DP.
  \emph{Sex among wild potatoes: ladies wear the pants.}
  Centre for Structural and Functional Genomics,
  Universitato Concordia, Montrealo, QC, Kanado;
  16 julio 2015.
  \\

\multicolumn{2}{c}{} \\

2014
& \textbf{Joly V} kaj Matton DP.
  \emph{Cell-cell communication between gametophytes and reproductive isolation
  in wild potatoes.}
  Dept. of Genome Sciences, Universitato de Vaŝingtonio, Seatlo, WA, Usono;
  24 aprilo 2014.
  \\

\multicolumn{2}{c}{} \\

2013
& \textbf{Joly V} kaj Matton DP.
  \emph{Species-specificity of pollen-pistil interactions in wild potatoes.}
  Instituto de Genetiko, Ĉina Akademio de Sciencoj, Pekino, Ĉinio;
  24 oktobro 2013. 
  \\

\end{tabularx}


% POSTER PRESENTATIONS
\section[Prezentoj kun afiŝo]{Prezentoj kun afiŝo
         \small en kongresoj kun kolega revizio \hfill {\mdseries\faStar}~Premio}

\begin{tabularx}{\textwidth}{@{}r|X@{}}

2018
& \textbf{Joly V} kaj Matton DP.
  \emph{Long-distance relationships: how the ovary perceives different
  pollination types at a distance.}
  Plant Biology 2018, Montrealo, QC, Kanado;
  14–18 julio 2018.
  \\

\multicolumn{2}{c}{} \\

2016
& \faStar~\textbf{Joly V}, Liu Y, Dorion S, Rivoal J kaj Matton DP.
  \emph{Ovule secretomics reveal the importance of post-transcriptional
  regulation of reproductive proteins.}
  Plant Reproduction 2016, Tusono, AZ, Usono;
  18–23 marto 2016.
  \vspace{1.5mm}
  \\

& \faStar~\textbf{Joly V} kaj Matton DP.
  \emph{KAPPA: exploring -omics data to detect and cluster cysteine-rich
  proteins.}
  [sama kongreso]
  \\

\multicolumn{2}{c}{} \\

2015
& \faStar~\textbf{Joly V} kaj Matton DP.
  \emph{KAPPA: meeting the challenge of proteome-wide detection and clustering
  of cysteine-rich proteins.}
  High Performance Computing Symposium HPCS 2015, Montrealo, QC, Kanado;
  17–19 junio 2015.
  \\

\multicolumn{2}{c}{} \\

2013
& \textbf{Joly V}, Liu Y kaj Matton DP.
  \emph{Interspecific divergence of reproductive proteins: the keystone of
  species-specific fertilization in wild potatoes?}
  10th Solanaceae Conference (SOL 2013), Pekino, Ĉinio;
  13–18 oktobro 2013.
  \vspace{1.5mm}
  \\

& \textbf{Joly V} kaj Matton DP.
  \emph{Speciation genes in pollen-pistil interactions.}
  9th Canadian Plant Genomics Workshop, Halifakso, NS, Kanado;
  12–15 aŭgusto 2013.
  \\

\end{tabularx}

\vspace{6mm}

% OTHER PRESENTATIONS
\section[Aliaj prezentoj]{Aliaj prezentoj \hfill \small{*Prezentisto}}

\begin{tabularx}{\textwidth}{@{}r|X@{}}

2018
& \textbf{Joly V} kaj Matton DP*.
  \emph{Pre-zygotic barriers in inter-specific crosses: a leading role for small
  cysteine-rich protein attractant in wild potatoes species ?}
  Plant Biology 2018, Montrealo, QC, Kanado;
  14–18 julio 2018.
  \\

\multicolumn{2}{c}{} \\

2017
& \textbf{Joly V} kaj Matton DP*.
  \emph{Pollination type recognition from a distance by the ovary is revealed
  by a global transcriptomic analysis.}
  5th International Symposium on Plant Signaling and Behavior, Matsue, Japanio;
  26 junio – 1 julio 2017.
  \\

\multicolumn{2}{c}{} \\

2013
& Liu Y*, Bai F, \textbf{Joly V} kaj Matton DP.
  \emph{Identification of female gametophyte-specific CRPs and isolation of
  pollen tube guidance attractant(s) in solanaceous species.}
  Journées du Centre SÈVE, Wendake, QC, Kanado;
  7–8 novembro 2013.
  \vspace{1.5mm}
  \\

& Tebbji F, \textbf{Joly V} kaj Matton DP*. \emph{Pollination type recognition
  from a distance by the ovary is revealed by a global transcriptomic analysis.}
  10th Solanaceae Conference (SOL 2013), Pekino, Ĉinio;
  13–18 oktobro 2013.
  \vspace{1.5mm}
  \\

& Liu Y*, \textbf{Joly V} kaj Matton DP.
  \emph{Isolation and characterization of the pollen tube attractant from}
  Solanum chacoense. [sama konferenco].
  \\

\multicolumn{2}{c}{} \\

2011
& Daigle C*, \textbf{Joly V} kaj Matton DP.
  \emph{Discovering new MAPK signalling cascades involved in plant reproduction
  using co-expression analyses and deep transcriptomic sequencing of ovule
  and pollen tubes.}
  7th Canadian Plant Genomics Workshop, Niagara Falls, ON, Kanado;
  22–25 aŭgusto 2011.
  \\

\end{tabularx}

\newpage

% TEACHING
\section{Instruado}

\begin{tabularx}{\textwidth}{@{}r|X@{}}

{\heavy Planta}
& {\heavy Ĉefasistanto,} {\bfseries 2013--2018} \\
{\heavy fiziologio}
& {\heavy Asistanto,} {\bfseries 2011--2012} \\
& {\em Planta fiziologio, praktikaj kursoj, P-ro Jean Rivoal Universitato de Montrealo}
  \vspace{1mm} \\
& •~Instrua ŝarĝo: 140 horoj por kvarono, ĉirkaŭ 80 lernantoj \\
& •~Semajna klasoj inkluzive de prelego (0:45) kaj praktika laboro (2:30) \\
& •~Superrigardo de 1--2 asistantoj \\

\multicolumn{2}{c}{} \\

\heavy{Molekula}
& {\heavy Asistanto,} {\bfseries 2014--2016} \\
\heavy{biologio}
& {\em Molekula biologio, praktikaj kursoj, P-ro Daniel P. Matton, Universitato de Montrealo}
  \vspace{1mm} \\
& •~Instrua ŝarĝo: 110 horoj por kvarono, 10–20 lernantoj \\
\end{tabularx}

\vspace{6mm}

\section{Superrigardo de staĝantoj}

\begin{tabularx}{\textwidth}{@{}r|llll@{}}
{\heavy Gradstudantoj}
 & \multicolumn{4}{X}{\small\em Ĉi tiuj latin-amerikaj studentoj estis gastigitaj en la laboratorio de mia profesoro kiel parto de la Programo de Futuraj Lideroj en la Amerikoj (PFLA-ELAP) de la Registaro de Kanado. Mi estis ilia superrigardisto por 5- al 6-monataj staĝoj rilatigitaj al mia esplorprojekto. \vspace{2mm}} \\
 & \textbf{• Kelly Rodrigues} & 2018-19 & Ph.D. & Univ. de San-Paŭlo (Brazilo) \\
 & \textbf{• Federico Ceriotti} & 2017-18 & M.Sc. & Nac. Univ. de Kujo (Argentino) \\
 & \textbf{• Carlos Bravo} & 2016-17 & Ph.D. & Nac. Univ. de Meksiko (Meksiko) \\
 & \textbf{• Laura González} & 2016 & Ph.D. & Nac. Univ. de Kordobo (Argentino) \\
 & \textbf{• Mariana Quiroga} & 2015 & Ph.D. & Nac. Univ. de Kordobo (Argentino) \\

\multicolumn{2}{c}{} \\

{\heavy B.Sc. studantoj}
 & \multicolumn{4}{X}{\small\em Mi superrigardis ĉi tiujn  studentojn por 4- al 6-monataj staĝoj necesaj por ilia bakalaŭra programo. \vspace{2mm}} \\
 & \textbf{• Maude Dorval} & 2018 & B.Sc. & Univ. de Montrealo (Kanado) \\
 & & 2017 & DEC & Kolegio Ahuntsic (Kanado) \\
 & \textbf{• Anna Zaro Sánchez} & 2017 & B.Sc. & Univ. de Barcelono (Hispanio) \\
 & \textbf{• Francis Banville} & 2017 & B.Sc. & Univ. de Montrealo (Kanado) \\
 & \textbf{• Andréa Davrinche} & 2014 &  B.Sc. & Univ. P. kaj M. Curie (Francio) \\
 & \textbf{• Ella Gangbe} &  2013 & B.Sc. & Univ. de Montrealo (Kanado) \\
 & \textbf{• Tissicca Hour} &  2012 & B.Sc. & Univ. de Montrealo (Kanado) \\
\end{tabularx}

\newpage


% SCHOLARSHIPS AND AWARDS
\section[Stipendioj kaj premioj]{Stipendioj kaj premioj
         \hfill \small{{\mdseries\faStar}~Grava stipendio aŭ premio}}

\begin{tabularx}{\textwidth}{@{}r|X@{}}

2018

& \textbf{Vojaĝ-stipendio “Jacques-Rousseau”} \\
& Plantbiologio Esplorinstituto, Universitato de Montrealo, 800~CAD \\

\multicolumn{2}{c}{} \\

2017

& \faStar~\textbf{Plejboneca stipendio “Hydro-Québec” (2-a jaro)} \\
& Hydro-Québec (nacia elektra kompanio), 25 000~CAD
  \vspace{1.3mm} \\

& \textbf{Stipendio por finantaj doktoriĝontoj (BFED)} \\
& Fakultato de Diplomiĝintaj kaj Postdoktoriĝaj Studoj, Univ. de Montrealo, 8 400~CAD
  \vspace{1.3mm} \\

& \textbf{Vojaĝ-stipendio “Jacques-Rousseau”} \\
& Plantbiologio Esplorinstituto, Universitato de Montrealo, 1 500~CAD
  \vspace{1.3mm} \\

& \textbf{Vojaĝ-stipendio (\emph{Bourse d'appui à la diffusion des résultats de recherche})} \\
& Fakultato de Diplomiĝintaj kaj Postdoktoriĝaj Studoj, Univ. de Montrealo, 500~CAD
  \vspace{1.3mm} \\

& \textbf{Honora mencio por studenta parola prezento} \\
& CSPB Eastern Regional Meeting \\

\multicolumn{2}{c}{} \\

2016

& \faStar~\textbf{Plejboneca stipendio “Hydro-Québec”} \\
& Hydro-Québec (nacia elektra kompanio), 25 000~CAD
  \vspace{1.3mm} \\

& \faStar~\textbf{Doktoriĝonta stipendio de la Registaro de Kebekio} \\
& Fonds Québécois de Recherche – Nature et Technologies, 13 333~CAD
  \vspace{1.3mm} \\

& \faStar~\textbf{Premio “MITACS Globalink” – Somera programo de la JSPS} \\
& MITACS / Japanese Society for the Promotion of Science, 534 000~JPY
  \vspace{1.3mm} \\

& \textbf{Premio de la plej bona gradigita studenta afiŝo} \\
& Frontiers in Plant Reproduction Biology, Konf. “Plant Reproduction 2016”, 300~USD
  \vspace{1.3mm} \\

& \textbf{Vojaĝ-stipendio “Jacques-Rousseau”} \\
& Plantbiologio Esplorinstituto, Universitato de Montrealo, 1 500~CAD
  \vspace{1.3mm} \\

& \textbf{Vojaĝ-subvencio “PARSECS”} \\
& FAÉCUM, Universitato de Montrealo, 400~CAD \\

\multicolumn{2}{c}{} \\

2015

& \faStar~\textbf{Plejboneca stipendio “Catherine-Frédette” pri biologio kaj neŭrologio} \\
& Fakultato de Diplomiĝintaj kaj Postdoktoriĝaj Studoj, Univ. de Montrealo, 5 000~CAD
  \vspace{1.3mm} \\

& \faStar~\textbf{Doktoriĝonta stipendio “FBSB” de la Departamento de Biologio} \\
& Universitato de Montrealo, 1 500~CAD
  \vspace{1.3mm} \\

& \textbf{Premio de la prezidanto por la plej bona studenta parola prezento} \\
& Kanada Societo de Plantaj Biologoj (CSPB-SCBV), Konferenco “Botany 2015”, 500~CAD
  \vspace{1.3mm} \\

& \textbf{Premio de la plej bona studenta afiŝo} \\
& Compute Canada, High Performance Computing Symposium HPCS 2015, 500~CAD
  \vspace{1.3mm} \\

& \textbf{Vojaĝ-stipendio “G.-H. Duff”} \\
& Kanada Societo de Plantaj Biologoj (CSPB-SCBV), 340~CAD
  \vspace{1.3mm} \\

& \textbf{Vojaĝ-stipendio “Jacques-Rousseau”} \\
& Plantbiologio Esplorinstituto, Universitato de Montrealo, 770~CAD
  \vspace{1.3mm} \\

& \faStar~\textbf{Plejboneca stipendio de la Fakultato de Diplomiĝintaj kaj Postdoktoriĝaj Studoj} \\
& Universitato de Montrealo, 3 000~CAD \\

\end{tabularx}

\section*{Stipendioj kaj premioj \small{(daŭrigo)}
          \hfill \small{{\mdseries\faStar}~Grava stipendio aŭ premio}}

\begin{tabularx}{\textwidth}{@{}r|X@{}}

2014

& \faStar~\textbf{Stipendio “Pehr-Kalm”} \\
& Botanika Ĝardeno de Montrealo, 2 000~CAD
  \vspace{1.3mm} \\

& \textbf{Vojaĝ-stipendio port internaciaj staĝoj} \\
& Registaro de Kebekio (FRQNT) – Centre SÈVE, 3 815~CAD
  \vspace{1.3mm} \\

& \textbf{Vojaĝ-stipendio “Jacques-Rousseau”} \\
& Plantbiologio Esplorinstituto, Universitato de Montrealo, 1 760~CAD \\

\multicolumn{2}{c}{} \\

2013

& \faStar~\textbf{Plejboneca stipendio “Marie-Victorin”} \\
& Plantbiologio Esplorinstituto, Universitato de Montrealo, 3 000~CAD
  \vspace{1.3mm} \\

& \textbf{Premio de la plej bona parola prezento} \\
& Konferenco “Journées du Centre SÈVE”, 300~CAD
  \vspace{1.3mm} \\

& \textbf{Vojaĝ-stipendio “Jacques-Rousseau”} \\
& Plantbiologio Esplorinstituto, Universitato de Montrealo, 850~CAD
  \vspace{1.3mm} \\

& \textbf{Premio de la plej bona parola prezento} \\
& Simpozio de biologio, Universitato de Montrealo, 100~CAD \\

\multicolumn{2}{c}{} \\

2012

& \faStar~\textbf{Majstra stipendio “FBSB” de la Departamento de Biologio} \\
& Universitato de Montrealo, 1 200~CAD
  \vspace{1.3mm} \\

& \faStar~\textbf{Stipendio por akcelita M.Sc.-al-Ph.D. transiro} \\
& Fakultato de Diplomiĝintaj kaj Postdoktoriĝaj Studoj, Univ. de Montrealo, 14 000~CAD \\

\multicolumn{2}{c}{} \\

2011

& \textbf{Vojaĝ-stipendio por studenta interŝanĝo en Kanado} \\
& Franca registaro (CROUS), 1 600~EUR
  \vspace{1.3mm} \\

& \textbf{Plejboneca stipendio “PIL” por studenta interŝanĝo en Kanado} \\
& Universitato Pierre-kaj-Marie-Curie (Parizo VI), 1 500~EUR
  \vspace{1.3mm} \\

& \textbf{Vojaĝ-stipendio “AMIÉ” por studenta interŝanĝo en Kanado} \\
& Franca regiona aŭtoritato (\emph{Conseil régional}), 2 800~EUR
  \vspace{1.3mm} \\

& \textbf{Vojaĝ-stipendio “Campus'Trotter” por studenta interŝanĝo en Kanado} \\
& Franca loka aŭtoritato (\emph{Conseil général}), 700~EUR \\

\multicolumn{2}{c}{} \\

2010

& \textbf{Plej bona bakalaŭra studento de biologio post la finaj ekzamenoj de junio 2010} \\
& Universitato Pierre-kaj-Marie-Curie (Parizo VI) \\

\multicolumn{2}{c}{} \\

2008

& \faStar~\textbf{Plejboneca stipendio por bakalaŭraj studoj} \\
& Franca registaro (CROUS), 5 400~EUR \\


\end{tabularx}

\newpage

% COMMITMENTS
\section{Devontigo}

\begin{tabularx}{\textwidth}{@{}r|X@{}}

{\heavy Societoj}

 & {\heavy Usona Societo de Plantaj Biologoj (ASPB),} {\bfseries ekde 2016}
   \vspace{2mm} \\

 & {\heavy Kanada Societo de Plantaj Biologoj (CSPB-SCBV),} {\bfseries ekde 2014}
   \vspace{2mm} \\

 & {\heavy Interacia Asocio por Esploro pri Seksa} \\
 & {\heavy Reproduktado de Plantoj (IASPRR),} {\bfseries ekde 2015}
   \vspace{2mm} \\

 & {\heavy Asocio de Biologoj Kebekiaj (ABQ),} {\bfseries 2013--2018}
   \vspace{2mm} \\

 & {\heavy Franca Societo de Botanikoj (SBF),} {\bfseries 2010--2011}
   \\

\multicolumn{2}{c}{} \\

{\heavy Studentaj}
  & {\heavy Asocio de Naturalista Studentoj \emph{Timarcha},} {\bfseries 2010--2011} \\
{\heavy asocioj}
  & Universitato Pierre-kaj-Marie-Curie (UPMC), Parizo, Francio
    \vspace{2mm} \\

  & {\heavy Ekologia komitato \emph{Éco-école},} {\bfseries 2006--2008} \\
  & Lycée Saint-Sauveur ($\approx$ mezlernejo), Redon, Francio \\

\multicolumn{2}{c}{} \\

{\heavy Voluntulado}

 & {\heavy Voluntula franca instruisto por enmigrintoj,} {\bfseries 2015--2016} \\
 & Komunuma Centro \emph{La Maison de l’Amitié}, Montrealo, QC, Kanado \\
 & •~3-horaj lecionoj ĉiun semajnon kun 10-20 studentoj
   \vspace{2mm} \\

 & {\heavy Kontribuanto al diversaj interretaj projektoj:} \\
 & •~Verkisto kaj tradukisto por \emph{Vikipedio}
   (artikoloj pri biologio), ekde 2008 \\
 & •~Volontula kartografiisto por \emph{OpenStreetMap}, ekde 2015 \\
 & •~Herbaria ciferecaĵo por la Pariza Nacia Muzeo
     de Natura Historio (Projekto “\emph{Les Herbonautes}”), 2015 \\

\end{tabularx}

\vspace{6mm}

% OTHER SKILLS
\section{Aliaj kompetentoj}

\begin{tabularx}{\textwidth}{@{}r|X@{}}

{\heavy Lingvoj}
& \textbf{Franca,} denaska lingvo \\
& \textbf{Angla,} flua \\
& \textbf{Hispana,} flua \\
& \textbf{Itala,} meza \\
& \textbf{Esperanto kaj Japana,} komencanto \\

\multicolumn{2}{c}{} \\

{\heavy Komputiko}
& \textbf{Programado:} Python kaj R. Baza nivelo de C kaj Perl.
  \vspace{2mm} \\

& \textbf{TTT:} HTML/CSS, Jekyll.
  \vspace{2mm} \\

& \textbf{Operaciumoj:} Linukso (\emph{Ubuntu}, \emph{Fedora},
  \emph{CentOS}), Mac OS X, Windows.
  \vspace{2mm} \\

& \textbf{Biokomputiko:} vic-rekonstruiloj (\emph{Trinity}, \emph{CLC}, etc.);
  liniigiloj por mallongaj vicoj (\emph{Bowtie}, \emph{TopHat}, etc.);
  vic-serĉiloj kaj liniigiloj (\emph{BLAST}, etc.);
  prinotiloj (\emph{BLAST2GO}, \emph{PFAMscan}, \emph{SignalP}, etc.)
  \vspace{2mm} \\

& \textbf{Oficejaj programaroj:} \LaTeX, \emph{LibreOffice}/\emph{OpenOffice},
  \emph{Microsoft Office}
  \vspace{2mm} \\

& \textbf{Prilaborado de bildoj:} \emph{GIMP}, \emph{Inkscape}, \emph{ImageJ},
  \emph{Adobe Photoshop}, \emph{Cytoscape} ; \emph{AxioVision} (stiradilo por mikroskopoj \emph{Zeiss}) \\

\end{tabularx}

\end{document}
