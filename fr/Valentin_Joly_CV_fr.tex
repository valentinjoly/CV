\documentclass[letterpaper,10pt]{article}

% MARGINS
\usepackage[
    top=2.5cm,
    bottom=0cm,
    outer=2.5cm,
    inner=2.5cm,
    heightrounded,
    marginparwidth=3cm,
    marginparsep=0.5cm]{geometry}


% LANGUAGE
\usepackage{polyglossia}
\setdefaultlanguage{french}


% FONTS
\usepackage{fontspec}
\defaultfontfeatures{Mapping=tex-text}
\setmainfont{Lato}
\usepackage{fontawesome}


% PAGE LAYOUT
\usepackage{parskip}
\usepackage{titlesec}
\titleformat{\section}{\Large\bfseries\raggedright}{}{0em}{}[\titlerule]
\titlespacing{\section}{0pt}{3pt}{3pt}


% COLOURS
\usepackage{xcolor}
\definecolor{linkcolour}{rgb}{0,0.2,0.6}


% HYPERLINKS
\usepackage{hyperref}
\hypersetup{
    colorlinks,
    breaklinks,
    urlcolor=linkcolour,
    linkcolor=linkcolour}


% DOCUMENT
\begin{document}
\pagestyle{empty}


% TITLE
\begin{center}
{\Huge Valentin Joly} \par
{\Large Biologiste moléculaire \textbullet{} Bioinformaticien}
\bigskip

% CONTACT DATA
\begin{tabular}{lll}
  \faPhoneSquare~+1 (438) 495-3215
  & \faEnvelopeSquare~\href{mailto:valentin.joly@gmail.com}{valentin.joly@gmail.com}
  & \faLinkedinSquare~\href{https://www.linkedin.com/in/valentinjoly}{valentinjoly} \\

  \faSkype~valentin.joly
  & \faExternalLinkSquare~\href{http://www.vjoly.net/fr/index.html}{http://www.vjoly.net}
  & \faGithub~\href{https://github.com/valentinjoly}{valentinjoly}
\end{tabular}
\end{center}
\bigskip


% EDUCATION
\section{Formation académique}
\begin{tabular}{r|p{15cm}}

  \textbf{Ph.D.}
  & \textbf{Sciences biologiques}, depuis 2013
    ~~~\small{(thèse à soumettre en déc. 2018)} \\

  \textbf{M.Sc.}
  & \textbf{Sciences biologiques}, 2012
    ~~~\small{(passage accéléré au doctorat en 2013)} \vspace{0.5mm} \\
  & \hspace{1.5mm} Université de Montréal, \emph{Montréal, QC, Canada} \\
  & \hspace{1.5mm} {\small \textbf{Thèse:} Communication cellulaire entre gamétophytes mâle et femelle} \\
  & \hspace{1.5mm} {\small \phantom{\textbf{Thèse:}} et maintien des barrières interspécifiques chez \emph{Solanum} sect. \emph{Petota}.} \\
  & \hspace{1.5mm} {\small \textbf{Directeur de recherche:} P\textsuperscript{r}~Daniel P. Matton}
    \hspace{1mm} \\

  \multicolumn{2}{c}{} \\

  \textbf{B.Sc.}
  & \textbf{Sciences du vivant, programme international}, 2011 \vspace{0.5mm} \\
  & \hspace{1.5mm} Université Pierre et Marie Curie (UPMC),
    \emph{Paris, France}: 1\textsuperscript{re} et 2\textsuperscript{e} années \\
  & \hspace{1.5mm} Université de Montréal (UdeM),
    \emph{Montréal, QC, Canada}: 3\textsuperscript{e} année en échange \\

\end{tabular}

\bigskip


% RESEARCH EXPERIENCE
\section[Expérience en recherche]{Expérience en recherche
         \hfill \small{{\mdseries\faFlask}~Biologie moléculaire~~~{\mdseries\faCode}~Bioinformatique}}

\begin{tabular}{r|p{14.5cm}}

\textbf{Canada}
& \textbf{Projet de doctorat,} depuis 2013.
  \hspace{1mm} \small{P\textsuperscript{r} D. P. Matton, Université de Montréal.}
  \vspace{0.5mm} \\
& \hspace{1.5mm} \faFlask~Manipulation d'ADN et ARN. Clonage. Expression et purification de protéines. \\
& \hspace{1.5mm} \faFlask~Cultures cellulaires végétales. Tests de guidage du tube pollinique. Microfluidique. \\
& \hspace{1.5mm} \faFlask~Microscopie: épifluorescence, confocal, MEB, MET. \\
& \hspace{1.5mm} \faCode~Programmation Python et R. Développement de l'outil de recherche de séquences KAPPA. \\
& \hspace{1.5mm} \faCode~Transcriptomique: Assemblages RNA-seq. Biopuces. DGE. Annotation fonctionnelle. \\
& \hspace{1.5mm} \faCode~Protéomique: Analyse de données LC-MS. Sécrétomique. Quantification \emph{label-free}. \\

\multicolumn{2}{c}{} \\

\textbf{Suède}
& \textbf{Collaboration internationale,} depuis mai 2016.
  \hspace{1mm} \small{D\textsuperscript{r} Johan Edqvist, Université de Linköping.}
  \vspace{0.5mm} \\
& \hspace{1.5mm} \faFlask~Expression et purification de protéines chez \emph{Pichia pastoris}. \\
& \hspace{1.5mm} \faCode~Développement d'un outil de prédiction et d'une base de données de nsLTP végétales. \\

\multicolumn{2}{c}{} \\

\textbf{Japon}
& \textbf{Programme d’été de la JSPS,} juin–août 2016.
  \hspace{1mm} \small{P\textsuperscript{r} T. Higashiyama, Université de Nagoya.}
  \vspace{0.5mm} \\
& \hspace{1.5mm} \faFlask~Développement de dispositifs microfluidiques pour l’étude du guidage des tubes polliniques. \\
& \hspace{1.5mm} \faFlask~Introduction à la microscopie confocale à deux photons. \\

\multicolumn{2}{c}{} \\

\textbf{États-Unis}
& \textbf{Stage international de recherche,} avr.–mai 2014.
  \hspace{1mm} \small{P\textsuperscript{r} W. J. Swanson, University of Washington.}
  \vspace{0.5mm} \\
& \hspace{1.5mm} \faCode~Détection de variants dans des données de séquençage de masse (GATK). \\
& \hspace{1.5mm} \faCode~Étude de l’évolution moléculaire des séquences (sélection positive) avec codeml. \\

\multicolumn{2}{c}{} \\

\textbf{Argentine}
& \textbf{Séjour botanique sur le terrain,} avr.–mai 2012.
  \hspace{1mm} \small{D\textsuperscript{r} F. Chiarini, Universidad Nacional de Córdoba.}
  \vspace{0.5mm} \\
& \hspace{1.5mm} \faFlask~Collecte d’individus de pommes de terre sauvages dans la cordillère des Andes. \\

\end{tabular}

\vspace{2mm}
\hfill \emph{\small voir page suivante}

\newpage

\section*{Expérience en recherche \small{(suite)}
          \hfill \small{{\mdseries\faFlask}~Biologie moléculaire~~~{\mdseries\faCode}~Bioinformatique}}
\begin{tabular}{r|p{13.5cm}}

\textbf{Canada}
& \textbf{Stage de recherche,} janv.–août 2011.
  \hspace{1mm} \small{P\textsuperscript{r} D. P. Matton, Université de Montréal.}
  \vspace{0.5mm} \\
& \hspace{1.5mm} \faFlask~Clonage moléculaire. Biolistique. Microscopie confocale et à épifluorescence. \\

\multicolumn{2}{c}{} \\

\textbf{France}
& \textbf{Stage de recherche,} juin–juill. 2010.
  \hspace{1mm} \small{P\textsuperscript{r} C. Bailly, CNRS/UPMC, Paris.}
  \vspace{0.5mm} \\
& \hspace{1.5mm} \faFlask~Physiologie de la dormance des semences
  \vspace{2.5mm} \\
& \textbf{Stage court d’initiation à la recherche,} janv. 2009.
  \hspace{1mm} \small{P\textsuperscript{r} C. Bowler, CNRS/ENS, Paris.}
  \vspace{0.5mm} \\
& \hspace{1.5mm} \faFlask~Électrophorèse de protéines. Immunoprécipitation. Western Blot. \\

\end{tabular}

\bigskip\bigskip


% EXTRA TRAINING
\section{Formation complémentaire}

\begin{tabular}{r|p{13.5cm}}

\textbf{Bioinformatique}
& \textbf{Spécialisation en ligne,} 2016--2018
  \hspace{1mm} \small{UC San Diego, sur Coursera} \\
& \textbullet{}~6 cours différents et un projet final: \hspace{0.5mm}
  Certificat
  \href{https://www.coursera.org/account/accomplishments/specialization/H528Q2K9KYB6}
  {H528Q2K9KYB6} \\

\multicolumn{2}{c}{} \\

\textbf{Python/R}
& \textbf{Cours en ligne de bioinformatique,} 2016
  \hspace{1mm} \small{Johns Hopkins University, sur Coursera} \\
& \textbullet{}~\emph{Python pour la science des données génomiques}: \hspace{0.5mm}
  Certificat
  \href{https://www.coursera.org/account/accomplishments/verify/XHKWDB4XD7}
  {XHKWDB4XD7} \\
& \textbullet{}~\emph{Introduction aux technologies génomiques}: \hspace{0.5mm}
  Certificat
  \href{https://www.coursera.org/account/accomplishments/verify/U88T89XKR2}
  {U88T89XKR2} \\
& \textbullet{}~\emph{Programmation en R}: \hspace{0.5mm}
  Certificat
  \href{https://www.coursera.org/account/accomplishments/verify/X8NKEQAUU4}
  {X8NKEQAUU4} \\

\multicolumn{2}{c}{} \\

\textbf{Annotation}
& \textbf{Séminaire international sur l’annotation fonctionnelle des protéines}, 2012 \\
\textbf{de séquences}
& BLAST2GO, University of California, Davis, CA, É.-U. \\

\end{tabular}

\bigskip
\bigskip

% PUBLICATIONS
\section[Publications]{Publications \hfill \small{*Contributions égales}}
\begin{tabular}{r|p{13.8cm}}

\textbf{Publiées}

& Salminen TA, Eklund DM, \textbf{Joly V}, Blomqvist K, Matton DP
  et Edqvist J. (2018).
  Deciphering the evolution and development of the cuticle by studying lipid
  transfer proteins in mosses and liverworts.
  \emph{Plants}, 7(1), 6.
  DOI: \href{http://doi.org/10.3390/plants7010006}{10.3390/plants7010006}
  \vspace{3mm} \\

& \textbf{Joly V} et Matton DP. (2015).
  KAPPA, a simple algorithm for the discovery and clustering of proteins defined by
  a key amino acid pattern.
  \emph{Bioinformatics}, 31(11), 1716--1723.
  DOI: \href{http://doi.org/10.1093/bioinformatics/btv047}
  {10.1093/bioinformatics/btv047}
  \vspace{3mm} \\

& Liu Y*, \textbf{Joly V*}, Dorion S, Rivoal J et Matton DP. (2015).
  The plant ovule secretome: a different view toward pollen-pistil interactions.
  \emph{Journal of Proteome Research}, 14(11):4763--75.
  DOI: \href{http://doi.org/10.1021/acs.jproteome.5b00618}
  {10.1021/acs.jproteome.5b00618}
  \vspace{3mm} \\

& Lafleur É*, Kapfer C*, \textbf{Joly V}, Liu Y, Tebbji F et coll. (2015).
  The ScFRK1 MAPK kinase kinase (MAPKKK) from \emph{Solanum chacoense} is
  involved in embryo sac and pollen development.
  \emph{Journal of Experimental Botany}, 66(7), 1833--1843.
  DOI: \href{http://doi.org/10.1093/jxb/eru524}{10.1093/jxb/eru524}
  \\

\multicolumn{2}{c}{} \\

\textbf{En préparation}

& \textbf{Joly V*}, Liu Y* et Matton DP.
  \emph{Solanum chacoense} ovule transcriptome reveals developmentally regulated
  transcripts during female gametophyte genesis and maturation.
  Soumission prévue en août 2018.
  \vspace{3mm} \\

& \textbf{Joly V*}, Tebbji F*, Nantel A et Matton DP.
  Pollination type recognition from a distance by the ovary is revealed by a
  global transcriptomic analysis.
  Soumission prévue en août 2018.
  \\

\end{tabular}


% ORAL PRESENTATIONS
\section[Présentations orales]{Présentations orales
                               \hfill \small{{\mdseries\faStar}~Prix}}
\begin{tabular}{r|p{15.1cm}}

2017

& \faStar~\textbf{Joly V}, Viallet C, Liu Y, Zaro A, Ceriotti F et Matton DP.
  \emph{Deciphering species-specific pollen tube guidance in \emph{Solanum}.}
  Rencontres régionales de l’Est du Canada, SCBV, Montréal, QC, Canada,
  24--25 nov. 2017.
  \vspace{1.5mm} \\

& \textbf{Joly V}, Viallet C, Liu Y et Matton DP.
  \emph{Reproductive cysteine-rich proteins: key players in \emph{Solanum}
  speciation?}
  Plant Biology 2017, Honolulu, HI, É.-U., 23--28 juin 2017. \\

\multicolumn{2}{c}{} \\

2016

& \textbf{Joly V} et Matton DP.
  \emph{Deciphering potatoes’ words of love.}
  Conférencier invité, Institute for Transformative bio-Molecules (ITbM),
  Université de Nagoya, Japon, 13 juill. 2016.
  \\

\multicolumn{2}{c}{} \\

2015

& \faStar~\textbf{Joly V} et Matton DP.
  \emph{Plants’ secret words of love: rapid evolution of pollen–pistil
  recognition proteins drives reproductive isolation of wild potatoes.}
  Botany 2015, Edmonton, AB, Canada, 26--19 juill. 2015.
  \vspace{1.5mm} \\

& \textbf{Joly V} et Matton DP.
  \emph{Sex among wild potatoes: ladies wear the pants.}
  Conférencier invité, Centre de Génomique Structurale et Fonctionnelle, Université Concordia, Montréal, QC, Canada, 16 juill. 2015.
  \\

\multicolumn{2}{c}{} \\

2014

& \textbf{Joly V} et Matton DP.
  \emph{Cell-cell communication between gametophytes and reproductive isolation
  in wild potatoes.}
  Conférencier invité, Dept. of Genome Sciences, University of Washington,
  Seattle, WA, É.-U., 24 avr. 2014.
  \\

\multicolumn{2}{c}{} \\

2013

& \faStar~\textbf{Joly V} et Matton DP.
  \emph{Comment éviter les liaisons dangereuses : secrets d’alcôve des pommes
  de terre.}
  Journées du Centre SÈVE, Wendake, QC, Canada, 7--8 nov. 2013.
  \vspace{1.5mm} \\

& \textbf{Joly V} et Matton DP.
  \emph{Species-specificity of pollen-pistil interactions in wild potatoes.}
  Conférencier invité, Institut de Génétique, Académie des Sciences de Chine,
  Pékin, Chine, 24 oct. 2013.
  \vspace{1.5mm} \\

& \faStar~\textbf{Joly V}, Liu Y et Matton DP.
  \emph{Divergence des protéines reproductives et maintien des barrières de
  spéciation chez les pommes de terre sauvages.}
  23\textsuperscript{e} Symposium des Sciences biologiques,
  Université de Montréal, Montréal, QC, Canada, 21 mars 2013.
  \\

\multicolumn{2}{c}{} \\

\end{tabular}


% POSTER PRESENTATIONS
\section[Présentations par affiche]{Présentations par affiche
                                    \hfill \small{{\mdseries\faStar}~Prix}}
\begin{tabular}{r|p{15.1cm}}

2018

& \textbf{Joly V} et Matton DP.
  \emph{Long-distance relationships: how the ovary perceives different
  pollination types at a distance.}
  Plant Biology 2018, Montréal, QC, Canada, 14--18 juill. 2018.
  \\

\multicolumn{2}{c}{} \\

2016

& \faStar~\textbf{Joly V}, Liu Y, Dorion S, Rivoal J et Matton DP.
  \emph{Ovule secretomics reveal the importance of post-transcriptional
  regulation of reproductive proteins.}
  Plant Reproduction 2016, Tucson, AZ, É.-U., 18--23 mars 2016.
  \vspace{1.5mm} \\

& \faStar~\textbf{Joly V} et Matton DP.
  \emph{KAPPA: exploring -omics data to detect and cluster cysteine-rich
  proteins.}
  [même conférence que ci-dessus]
  \\

\multicolumn{2}{c}{} \\

2015

& \faStar~\textbf{Joly V} et Matton DP.
  \emph{KAPPA: meeting the challenge of proteome-wide detection and clustering
  of cysteine-rich proteins.}
  High Performance Computing Symposium HPCS 2015, Montréal, QC, Canada,
  17--19 juin 2015.
  \\

\multicolumn{2}{c}{} \\

2013

& \textbf{Joly V}, Liu Y et Matton DP.
  \emph{Interspecific divergence of reproductive proteins: the keystone of
  species-specific fertilization in wild potatoes?}
  10th Solanaceae Conference (SOL 2013), Pékin, Chine, 13--17 oct. 2013.
  \vspace{1.5mm} \\

& \textbf{Joly V} et Matton DP.
  \emph{Speciation genes in pollen-pistil interactions.}
  9th Canadian Plant Genomics Workshop, Halifax, NS, Canada,
  12--15 août 2013.
  \\

\end{tabular}

\bigskip
\bigskip

% OTHER PRESENTATIONS
\section[Autres présentations]{Autres présentations
                               \hfill \small{*Personne en charge}}
\begin{tabular}{r|p{15.1cm}}

2018

& \textbf{Joly V} et Matton DP*.
  \emph{Pre-zygotic barriers in inter-specific crosses: a leading role for small
  cysteine-rich protein attractant in wild potatoes species ?}
  Plant Biology 2018, Montréal, QC, Canada, 14--18 juill. 2018.
  \\

\multicolumn{2}{c}{} \\

2017

& \textbf{Joly V} et Matton DP*.
  \emph{Pollination type recognition from a distance by the ovary is revealed
  by a global transcriptomic analysis.}
  5th International Symposium on Plant Signaling and Behavior, Matsue, Japan,
  26 juin -- 1\textsuperscript{er} juill. 2017.
  \\

\multicolumn{2}{c}{} \\

2013

& Liu Y*, Bai F, \textbf{Joly V} et Matton DP.
  \emph{Identification of female gametophyte-specific CRPs and isolation of
  pollen tube guidance attractant(s) in solanaceous species.}
  Journées du Centre SÈVE, Wendake, QC, Canada, 7--8 nov. 2013.
  \vspace{1.5mm} \\

& Tebbji F, \textbf{Joly V} et Matton DP*. \emph{Pollination type recognition
  from a distance by the ovary is revealed by a global transcriptomic analysis.}
  10th Solanaceae Conference (SOL 2013), Pékin, Chine, 13--17 oct. 2013.
  \vspace{1.5mm} \\


& Liu Y*, \textbf{Joly V} et Matton DP.
  \emph{Isolation and characterization of the pollen tube attractant from}
  Solanum chacoense. 10th Solanaceae Conference (SOL 2013), Pékin, Chine,
  13--17 oct. 2013. \\

\multicolumn{2}{c}{} \\

2011

& Daigle C*, \textbf{Joly V} et Matton DP.
  \emph{Discovering new MAPK signalling cascades involved in plant reproduction
  using co-expression analyses and deep transcriptomic sequencing of ovule
  and pollen tubes.}
  7th Canadian Plant Genomics Workshop, Niagara Falls, ON, Canada,
  22--25 août 2011.
  \\

\end{tabular}

\bigskip
\bigskip

% TEACHING
\section{Enseignement}
\begin{tabular}{r|p{14cm}}

\textbf{Physiologie}
& \textbf{Assistant d’enseignement en chef,} depuis 2013 \\
\textbf{végétale}
& \textbf{Assistant d’enseignement,} 2011--2012 \\
& TP d’introduction à la physiologie végétale, P\textsuperscript{r}~Jean Rivoal, Université de Montréal
  \vspace{1mm} \\
& \textbullet{} Charge: 140 heures par session, environ 70 étudiants \\
& \textbullet{} Séances hebdomadaires incluant un laïus (45 min) et des travaux
  pratiques \\
& \textbullet{} Encadrement de 1 à 2 auxiliaires d'enseignement \\

\multicolumn{2}{c}{} \\

\textbf{Biologie}
& \textbf{Assistant d’enseignement,} 2014--2016 \\
\textbf{moléculaire}
& TP de biologie moléculaire : ADN et ARN, P\textsuperscript{r}~D. P. Matton,
  Université de Montréal
  \vspace{1mm} \\
& \textbullet{}~Enseignement de travaux pratiques à un groupe de 10 à 20
  étudiants \\

\multicolumn{2}{c}{} \\

\textbf{Supervision}
& \textbf{Superviseur de stagiaires internationaux,
  niveaux M.Sc. et Ph.D.,} depuis 2015 \\
& Programme des Futurs Leaders dans les Amériques (PFLA-ELAP)
  à l’Université de Montréal
  \vspace{1mm} \\
& \textbullet{}~4 étudiants supervisés jusqu’à présent, pour des tages de 4 à 6
  mois
\vspace{2.5mm} \\

& \textbf{Superviseur de  stagiaires d’été, niveau B.Sc.}, depuis 2012 \\
& Cours d’initiation à la recherche, Université de Montréal
  \vspace{1mm} \\
& \textbullet{}~6 étudiants au premier cycle supervisés jusqu’à présent, pour
  des projets de 1 à 4 mois \\

\end{tabular}

\newpage


% SCHOLARSHIPS AND AWARDS
\section[Prix et bourses]{Prix et bourses
         \hfill \small{{\mdseries\faStar}~Bourse ou prix important}}
\begin{tabular}{r|p{14cm}}

2018

& \textbf{Bourse de voyage Jacques-Rousseau} \\
& Institut de Recherche en Biologie Végétale, Université de Montréal, 800~CAD \\

\multicolumn{2}{c}{} \\

2017

& \faStar~\textbf{Bourse d'excellence Hydro-Québec} \\
& Hydro-Québec (compagnie nationale d'électricité), 25~000~CAD
  \vspace{1.3mm} \\

& \textbf{Bourse de fin d'études doctorales (5\textsuperscript{e} année)} \\
& Faculté des Études Supérieures et Postdoctorales, Université de Montréal, 12~000~CAD \\

& \textbf{Bourse de voyage Jacques-Rousseau} \\
& Institut de Recherche en Biologie Végétale, Université de Montréal, 1~500~CAD
  \vspace{1.3mm} \\

& \textbf{Bourse d'appui à la diffusion des résultats de recherche} \\
& Faculté des Études Supérieures et Postdoctorales, Université de Montréal, 500~CAD \\

& \textbf{Mention honorale} pour une présentation orale étudiante \\
& Rencontres régionales de l’Est du Canada, SCBV \\

\multicolumn{2}{c}{} \\

2016

& \faStar~\textbf{Bourse d'excellence Hydro-Québec} \\
& Hydro-Québec (compagnie nationale d'électricité), 25~000~CAD
  \vspace{1.3mm} \\

& \faStar~\textbf{Prix MITACS-JSPS pour stage international Canada-Japon} \\
& MITACS -- Société Japonaise pour la Promotion de la Science, 550~000~JPY
  \vspace{1.3mm} \\

& \faStar~\textbf{Bourse de recherche de 3\textsuperscript{e} cycle} \\
& Fonds de Recherche du Québec -- Nature et Technologies, 13~333~CAD
  \vspace{1.3mm} \\

& \textbf{Prix du meilleur poster étudiant} \\
& Frontiers in Plant Reproduction Biology, Conférence \emph{Plant Reproduction 2016}, 300~USD
  \vspace{1.3mm} \\

& \textbf{Bourse de voyage Jacques-Rousseau} \\
& Institut de Recherche en Biologie Végétale, Université de Montréal, 1~500~CAD
  \vspace{1.3mm} \\

& \textbf{Subvention de voyage PARSECS} \\
& FAÉCUM, Université de Montréal, 400~CAD \\

\multicolumn{2}{c}{} \\

2015

& \faStar~\textbf{Bourse d'excellence Catherine-Fradette en sciences biologiques et neurologie} \\
& Faculté des Études Supérieures et Postdoctorales, Université de Montréal, 5~000~CAD
  \vspace{1.3mm} \\

& \textbf{Bourse du Fonds de Bourses en Sciences Biologiques (FBSB), niveau doctorat} \\
& Université de Montréal, 1~500~CAD
  \vspace{1.3mm} \\

& \textbf{Prix du Président pour la meilleure présentation orale étudiante} \\
& Société Canadienne de Biologie Végétale (SCBV), Conférence Botany 2015, 500~CAD
  \vspace{1.3mm} \\

& \textbf{Prix du meilleur poster étudiant} \\
& Calcul Canada, Symposium de calcul informatique de pointe HPCS 2015, 500~CAD
  \vspace{1.3mm} \\

& \textbf{Bourse de voyage G.-H. Duff} \\
& Société Canadienne de Biologie Végétale (SCBV), 340~CAD
  \vspace{1.3mm} \\

& \textbf{Bourse de voyage Jacques-Rousseau} \\
& Institut de Recherche en Biologie Végétale, Université de Montréal, 775~CAD
  \vspace{1.3mm} \\

& \faStar~\textbf{Bourse au mérite de la Faculté des Études Supérieures et Postdoctorales} \\
& Université de Montréal, 3~000~CAD\\

\multicolumn{2}{c}{} \\

2014

& \textbf{Bourse Pehr-Kalm} \\
& Jardin botanique de Montréal, 2~000~CAD
  \vspace{1.3mm} \\

& \faStar~\textbf{Bourse pour stagiaires internationaux} \\
& Fonds de Recherche du Québec en Nature et Technologies -- Centre SÈVE, 3~815~CAD \\

\end{tabular}

\vspace{3mm}
\hfill \emph{voir page suivante}

\section*{Prix et bourses \small{(suite)}
          \hfill \small{{\mdseries\faStar}~Bourse ou prix important}}
\begin{tabular}{r|p{14cm}}

2014

& \textbf{Bourse de voyage Jacques-Rousseau} \\
& Institut de Recherche en Biologie Végétale, Université de Montréal, 1~769~CAD\\

\multicolumn{2}{c}{} \\

2013

& \faStar~\textbf{Bourse d'excellence Marie-Victorin} \\
& Institut de Recherche en Biologie Végétale, Université de Montréal, 3~000~CAD
  \vspace{1.3mm} \\

& \textbf{Prix de la meilleure présentation orale} \\
& Journées du Centre SÈVE, 300~CAD
  \vspace{1.3mm} \\

& \textbf{Bourse de voyage Jacques-Rousseau} \\
& Institut de Recherche en Biologie Végétale, Université de Montréal, 850~CAD
  \vspace{1.3mm} \\

& \textbf{Prix de la meilleure présentation orale} \\
& Symposium de biologie de l'Université de Montréal, 100~CAD\\

\multicolumn{2}{c}{} \\

2012

& \textbf{Bourse du Fonds de Bourses en Sciences Biologiques (FBSB), niveau maîtrise} \\
& Université de Montréal, 1~200~CAD
  \vspace{1.3mm} \\

& \faStar~\textbf{Bourse de passage accéléré maîtrise-doctorat} \\
& Faculté des Études Supérieures et Postdoctorales, Université de Montréal, 14~000~CAD \\

\multicolumn{2}{c}{} \\

2011

& \textbf{Bourse de voyage pour échange au Québec (Complément mobilité CROUS)} \\
& Ministère français de l'Enseignement Supérieur et de la Recherche, 1~600~EUR
  \vspace{1.3mm} \\

& \textbf{Bourse d'excellence PIL pour échange au Québec} \\
& Université Pierre et Marie Curie (Paris VI), 1~500~EUR
  \vspace{1.3mm} \\

& \textbf{Bourse de voyage AMIÉ pour échange au Québec} \\
& Conseil régional d'Île-de-France (autorités régionales en France), 2~800~EUR
  \vspace{1.3mm} \\

& \textbf{Bourse de voyage Campus'Trotter pour échange au Québec} \\
& Conseil départemental du Morbihan (autorités locales en France), 700~EUR\\

\multicolumn{2}{c}{} \\

2010

& \textbf{Meilleur étudiant aux examens de licence en biologie} \\
& Université Pierre et Marie Curie (Paris VI), semestre S4 \\

\multicolumn{2}{c}{} \\

2008

& \faStar~\textbf{Bourse au mérite CROUS} \\
& Ministère français de l'Enseignement Supérieur et de la Recherche, 5~400~EUR \\

\end{tabular}

\bigskip
\bigskip

% COMMITMENTS
\section{Engagements}

\begin{tabular}{r|p{14cm}}

\textbf{Sociétés}

 & \textbf{American Society of Plant Biologists (ASPB),} depuis 2016
   \vspace{2mm} \\

 & \textbf{Société Canadienne de Biologie Végétale (CSPB-SCBV),} depuis 2014
   \vspace{2mm} \\

 & \textbf{International Association of Sexual Plant} \\
 & \textbf{Reproduction Research (IASPRR),} depuis 2015
   \vspace{2mm} \\

 & \textbf{Association des Biologistes du Québec} (ABQ), 2013--2018
   \vspace{2mm} \\

 & \textbf{Société Botanique de France} (SBF), 2010--2011 \\

\multicolumn{2}{c}{} \\

\textbf{Associations}
  & \textbf{Association naturaliste \emph{Timarcha}}, 2010--2011 \\
\textbf{étudiantes}
  & Université Pierre et Marie Curie (UPMC), Paris, France
    \vspace{2mm} \\

  & \textbf{Comité \emph{Éco-école}} d’actions pour l’environnement, 2006--2008 \\
  & Lycée Saint-Sauveur, Redon, France \\

\end{tabular}

\vspace{3mm}
\hfill \emph{\small voir page suivante}

\section*{Engagements \small{(suite)}}
\begin{tabular}{r|p{13.5cm}}

\textbf{Bénévolat}

 & \textbf{Enseignant de français pour des nouveaux arrivants au Canada,} 2015--2016 \\
 & Centre communautaire \emph{La Maison de l’Amitié}, Montréal, QC, Canada \\
 & \textbullet{}~Cours hebdomadaires de 3 h avec 10 à 20 étudiants
   \vspace{2mm} \\

 & \textbf{Contributeur à plusieurs projets en ligne:} \\
 & \textbullet{}~Rédacteur et traducteur pour \emph{Wikipedia}
   (articles de biologie), depuis 2008 \\
 & \textbullet{}~Cartographe bénévole pour \emph{OpenStreetMap},
   since 2015 \\
 & \textbullet{}~Contributeur au projet «~\emph{Les Herbonautes}~» visant à numériser l’herbier du Muséum National d’Histoire Naturelle de Paris, 2015 \\

\multicolumn{2}{c}{} \\

\textbf{Vulgarisation}

& \textbf{Entrevue radiophonique} pour l’émission scientifique de Radio-Canada,
  \href{http://ici.radio-canada.ca/emissions/les_annees_lumiere/2009-2010/chronique.asp?idChronique=404672}{\emph{Les années lumière}}.
  Diffusé le 24 avril 2016.
  \vspace{2mm} \\

& \textbf{Article de vulgarisation} rédigé pour
  \href{http://arnmessager.com/2014/12/19/les-mots-damour-des-plantes-a-fleurs/}
  {\emph{L'ARN messager}}, journal en ligne des étudiants en biologie de l’Université de Montréal. Publié le 19 déc. 2014. \\

\end{tabular}

\bigskip
\bigskip


% OTHER SKILLS
\section{Autres compétences}
\begin{tabular}{r|p{13cm}}

\textbf{Langues}
& \textbf{Français,} langue maternelle \\
& \textbf{Anglais,} courant \\
& \textbf{Espagnol,} courant \\
& \textbf{Italien,} intermédiaire \\
& \textbf{Japonais,} débutant \\

\multicolumn{2}{c}{} \\

\textbf{Computing}

& \textbf{Programmation:} Python et R. Bases en C et Perl.
  \vspace{2mm} \\

& \textbf{Internet:} HTML/CSS, Jekyll.
  \vspace{2mm} \\

& \textbf{Systèmes d’exploitation:} Linux (\emph{Ubuntu}, \emph{Fedora},
  \emph{CentOS}), Mac OS X, Windows.
  \vspace{2mm} \\

& \textbf{Bioinformatique:} assembleurs (\emph{Trinity}, \emph{CLC}, etc.);
  aligneurs (\emph{Bowtie}, \emph{TopHat}, etc.);
  outils de recherche et d’alignement de séquences (\emph{BLAST}, etc.);
  annotateurs (\emph{BLAST2GO}, \emph{PFAMscan}, \emph{SignalP}, etc.)
  \vspace{2mm} \\

& \textbf{Bureautique:} \LaTeX, \emph{LibreOffice}/\emph{OpenOffice},
  \emph{Microsoft Office}
  \vspace{2mm} \\

& \textbf{Traitement d’images:} \emph{GIMP}, \emph{Inkscape}, \emph{ImageJ},
  \emph{Adobe Photoshop}, \emph{Cytoscape} ; \emph{AxioVision} (logiciel de commande des microscopes \emph{Zeiss}) \\

\end{tabular}

\end{document}
